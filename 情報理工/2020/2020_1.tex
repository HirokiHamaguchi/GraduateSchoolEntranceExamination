\documentclass[a4paper, 10pt, dvipdfmx]{jlreq}

\usepackage{ascmac}
\usepackage{amsmath,amsfonts,amssymb}
\usepackage{bm}
\usepackage{mathtools}
\usepackage{siunitx}
\usepackage[dvipdfmx]{graphicx}
\usepackage[dvipdfmx]{color}
\usepackage[dvipdfmx, colorlinks=true, allcolors=blue]{hyperref}
\usepackage{listings, jlisting}
\usepackage{tikz}
\usepackage{physics}
\usepackage{url}

\Urlmuskip=0mu plus 10mu
\allowdisplaybreaks[4]
\frenchspacing
\definecolor{OliveGreen}{rgb}{0.0,0.6,0.0}
\definecolor{Orenge}{rgb}{0.89,0.55,0}
\definecolor{SkyBlue}{rgb}{0.28, 0.28, 0.95}
\lstset{
  language={c++},
  basicstyle={\ttfamily},
  identifierstyle={\small},
  ndkeywordstyle={\small},
  frame=single,
  breaklines=true,
  numbers=left,
  xrightmargin=0zw,
  xleftmargin=3zw,
  numberstyle={\scriptsize},
  lineskip=-0.9ex,
  keywordstyle={\small\bfseries\color{SkyBlue}},  
  commentstyle={\color{OliveGreen}}, 
  stringstyle={\small\ttfamily\color{Orenge}}    
}

\begin{document}

\title{2020年度 大問1}
\author{hari64boli64 (hari64boli64@gmail.com)}
\date{\today}
\maketitle


\section{問題}

\begin{align*}
    BC+CB=A \quad (\mathrm{where} A,B \in \mathbb{R}^{n\times n}, B\succ 0, B^T=B)
\end{align*}

\section{解答}

\subsection*{(1)}

$B=QDQ^T \quad (\mathrm{where} \; D \; \mathrm{is}\; \mathrm{diagonal}, Q^TQ=I)$とする。

\begin{align*}
                    & QDQ^TC+CQDQ^T=A             \\
    \Leftrightarrow & Q^TQDQ^TCQ+Q^TCQDQ^TQ=Q^TAQ \\
    \Leftrightarrow & DQ^TCQ+Q^TCQD=Q^TAQ         \\
    \Leftrightarrow & DC'+C'D=A'                  \\
\end{align*}

よって、対角行列のみを考えれば良い。

\begin{align*}
                    & A_{ij}=2b_{ii}c_{ij}                                           \\
    \Leftrightarrow & c_{ij}=\frac{A_{ij}}{2b_{ii}}\quad (\because \; b_{ii} \geq 0) \\
\end{align*}

よって、$C$は一意。

\subsection*{(2)}

$BC_{A,B}=C_{A,B}B \Rightarrow AB=BA$は自明。

$AB=BA \Rightarrow BC_{A,B}=C_{A,B}B$は

\begin{align*}
                    & AB=BA                                                             \\
    \Leftrightarrow & CBB=BBC                                                           \\
    \Leftrightarrow & CQD^2Q^T=QD^2Q^TC                                                 \\
    \Leftrightarrow & \sum_k (CQ)_{ik}D^2_{kk}Q^T_{kj}=\sum_k Q_{ik}D^2_{kk}(Q^TC)_{kj} \\
    \Leftrightarrow & \sum_k (CQ)_{ik}D_{kk}Q^T_{kj}=\sum_k Q_{ik}D_{kk}(Q^TC)_{kj}     \\
    \Leftrightarrow & CQDQ^T=QDQ^TC                                                     \\
    \Leftrightarrow & CB=BC
\end{align*}

より、示される。

\section{知識}

正定値性は、基本的には対称行列かエルミート行列(つまり、自己随伴行列、$A=A^{\dagger} \Leftrightarrow A^T=\overline{A}$)のみに限って考えることが多い。

実対称行列の正定値性は以下の4条件が同値。

\begin{itembox}[l]{実対称行列の正定値性の4条件}
    1. $x^TAx>0 \quad (\forall x\in \mathbb{R}^n\setminus {\boldsymbol{0}})$

    2. Aの固有値が全て正 (固有ベクトルに対して、$\langle Ax,x \rangle = \langle \lambda x,x \rangle =\lambda |x||^2$、逆も3より従う)

    3. ある直交行列$Q(Q^TQ=QQ^T=I)$と、対角成分が全て正の対角行列$D$を用いて、$A=QDQ^T$と表せる

    4. ある正則行列$S$を用いて、$A=S^TS$と表せる
\end{itembox}


以下の性質が上記の証明で重要。

\begin{itembox}[l]{$A=A^T$ならば、$P^{-1}AP$と対角化可能}

    ただし、$P$は直交行列で、$P^TP=PP^T=I$

    (1) 固有値がすべて異なる場合、固有ベクトル$\{p_k\}$は自動的に直交するので、大きさが1になるように選ぶことにより$(r_k=\frac{1}{|p_k|}p_k)$、
    $R=[r_1, r_2, \cdots, r_n]$は直交行列となり、この$R$を用いて、$R^{-1}AR$を対角行列にできる。

    (2) 固有値に重複がある場合にも、対称行列では、重複する固有値に属する1次独立な固有ベクトルを重複度分だけ見つけることが常に可能
    それらをグラム・シュミットの直交化法により正規直交化すれば、他の固有ベクトルも合わせてすべての固有ベクトルが直交することとなり、 それらを規格化すればやはり直交行列$R$が得られる。
\end{itembox}

また、正則であり、逆行列も正定値行列。

\begin{align*}
    A^{-1}=(UDU^-1)^{-1}=UD^{-1}U^{-1}
\end{align*}

ちなみに、本問はシルベスター方程式というものらしい。


\begin{thebibliography}{9}
    \bibitem{site:1}
    東京大学."建築数理設計特論 講義資料".\url{https://www.or.mist.i.u-tokyo.ac.jp/kanno/lecture/math_design/psd170420.pdf}
    \bibitem{site:2}
    数学の景色.“正定値行列・半正定値行列の定義・性質3つとその証明”.2022年8月9日.\url{https://mathlandscape.com/positive-definite-matrix/}
    \bibitem{site:3}
    武内修@筑波大."線形代数I/実対称行列の対角化".2023年3月14日.\url{https://dora.bk.tsukuba.ac.jp/~takeuchi/?%E7%B7%9A%E5%BD%A2%E4%BB%A3%E6%95%B0%EF%BC%A9%2F%E5%AE%9F%E5%AF%BE%E7%A7%B0%E8%A1%8C%E5%88%97%E3%81%AE%E5%AF%BE%E8%A7%92%E5%8C%96#rf85899c}
    \bibitem{site:4}
    Wikipedia."シルベスター方程式".\url{https://ja.wikipedia.org/wiki/%E3%82%B7%E3%83%AB%E3%83%99%E3%82%B9%E3%82%BF%E3%83%BC%E6%96%B9%E7%A8%8B%E5%BC%8F}
\end{thebibliography}
\end{document}
