\documentclass[a4paper, 10pt, dvipdfmx]{jlreq}

\usepackage{amsmath,amsfonts,amssymb}
\usepackage{bm}
\usepackage{mathtools}
\usepackage{siunitx}
\usepackage[dvipdfmx]{graphicx}
\usepackage[dvipdfmx]{color}
\usepackage[dvipdfmx, colorlinks=true, allcolors=blue]{hyperref}
\usepackage{listings, jlisting}
\usepackage{tikz}
\usepackage{physics}
\usepackage{url}

\Urlmuskip=0mu plus 10mu
\allowdisplaybreaks[4]
\frenchspacing
\definecolor{OliveGreen}{rgb}{0.0,0.6,0.0}
\definecolor{Orenge}{rgb}{0.89,0.55,0}
\definecolor{SkyBlue}{rgb}{0.28, 0.28, 0.95}
\lstset{
  language={c++},
  basicstyle={\ttfamily},
  identifierstyle={\small},
  ndkeywordstyle={\small},
  frame=single,
  breaklines=true,
  numbers=left,
  xrightmargin=0zw,
  xleftmargin=3zw,
  numberstyle={\scriptsize},
  lineskip=-0.9ex,
  keywordstyle={\small\bfseries\color{SkyBlue}},  
  commentstyle={\color{OliveGreen}}, 
  stringstyle={\small\ttfamily\color{Orenge}}    
}

\begin{document}

\title{2020年度 大問5}
\author{hari64boli64 (hari64boli64@gmail.com)}
\date{\today}
\maketitle


\section{問題}

近似配列

\begin{align*}
    \min_{B} \sum_{i=1}^{n}{(A[i]-B[i])^2}
\end{align*}

$$
    B[1]\leq B[2] \leq \cdots \leq B[n]
$$

\section{解答}

まず、これは問題に誤りがあると思われる。例えば

\begin{align*}
    A   & =[0,-1,0,1,0]  \\
    B_1 & =[-1,-1,0,1,1] \\
    B_2 & =[0,0,0,0,0]   \\
    (A' & =A+[100])
\end{align*}

などとすると、近似配列に一意性は無いことが分かるが、そのようなことが(1-1)では考慮されていない様に見受けられる。

以下、この議論は省略する。

\subsection*{(1)}

\subsubsection*{(1-1)}

$B'[n+1] \neq A'[n+1]$を仮定し、大小関係で場合分けをする。

\begin{quote}
    \begin{itemize}
        \item  $B'[n+1] < A'[n+1]$

              $n$番目以下に関して、実行可能領域が狭まるので、解は悪化する。
              $n+1$番目に関しても悪化。
              よって、全体で悪化しており、これは最適解にはならない。

        \item  $B'[n+1] > A'[n+1]$

              $B'[n+1]>B[n]$より、$B'[n+1]$の値を小さくすれば改善される。
              よって、これは最適解にはならない。
    \end{itemize}
\end{quote}


\subsubsection*{(1-2)}

問題で与えられている近似配列について、$B[i]=b$と置く。

$B'[n+1]<B[n]$を以下では仮定する。

$B[1]=b_1$という変数についてのみの制約付き最適化問題を考える。
この時、$\min_{b_1\leq b}(A[1]-b_1)^2$という制約付き最適化問題を考えると、近似配列において$B[1]=b$であることから、この部分問題においても$b_1=b$が最適解であると分かる。特に、目的関数が凸であるため、$b_1 \leq b$全体において、+に変化させる方が目的関数の値を減少させると分かる。

いま、$B'[1]=b'_1$に関して、$\min_{b'_1 \leq B'[2] (\leq b)}(A[1]-b'_1)^2$という部分的な制約付き最適化問題を考えると、先の議論より、$b'_1=B'[1]=B'[2]$が最適解である。

続いて、同様に、$B[1]=B[2]=b_2$についても、部分的な制約付き最適化問題を考える。

近似配列の形から、また、$\sum_{i=1}^{2}(A[i]-b_2)^2$が凸であることから、同様の議論が行え、$B'[1]=B'[2]=b'_2$に関して、$\min_{b'_2\leq B'[3](\leq b)}\sum_{i=1}^{2}(A[i]-b'_2)^2$の最適解は$b'_2=B'[1]=B'[2]=B'[3]$である。

以下、帰納的に考えると、$B'[1]=B'[2]=\cdots=B'[n]=B'[n+1]$が言える。

あとは、この条件の元での最適解が以下で示す形であることから、前半の題意は直ちに従う。

$B'[i]=b \; (\forall i)$とすると、

\begin{align*}
    \sum_{i=1}^{n}{(A[i]-b)^2} = \sum_{i=1}^{n}{A[i]^2} -2b \sum_{i=1}^{n}{A[i]} + nb^2
\end{align*}

よって、$b=\frac{1}{n}\sum_{i=1}^{n}{A[i]}$が最適解。

\subsection*{(2)}

dpをする。

k番目までの近似配列がそれぞれ一つ求まっているとする。

\begin{quote}
    \begin{itemize}
        \item  $A[k+1] \geq B[k]$

              (1-1)より、$B[k+1]=A[k+1]$で最適

        \item  $A[k+1] < B[k]$

              $B[k]-B[i] \; (\forall 1 \leq i \leq k)$だけ、全体をずらすと、(1-2)に帰着される。
              よって、近似配列が定まる。
              (記述は難しい。ここがある意味本質だと思うが、図を書いた方が分かりやすいので、ここでは省略する)
    \end{itemize}
\end{quote}

これは明らかに多項式時間で求まる。

\section{知識}

特になし

\end{document}