\documentclass[a4paper, 10pt, dvipdfmx]{jlreq}

\usepackage{ascmac}
\usepackage{amsmath,amsfonts,amssymb}
\usepackage{bm}
\usepackage{mathtools}
\usepackage{siunitx}
\usepackage[dvipdfmx]{graphicx}
\usepackage[dvipdfmx]{color}
\usepackage[dvipdfmx, colorlinks=true, allcolors=blue]{hyperref}
\usepackage{listings, jlisting}
\usepackage{tikz}
\usepackage{physics}
\usepackage{url}

\Urlmuskip=0mu plus 10mu
\allowdisplaybreaks[4]
\frenchspacing
\definecolor{OliveGreen}{rgb}{0.0,0.6,0.0}
\definecolor{Orenge}{rgb}{0.89,0.55,0}
\definecolor{SkyBlue}{rgb}{0.28, 0.28, 0.95}
\lstset{
  language={c++},
  basicstyle={\ttfamily},
  identifierstyle={\small},
  ndkeywordstyle={\small},
  frame=single,
  breaklines=true,
  numbers=left,
  xrightmargin=0zw,
  xleftmargin=3zw,
  numberstyle={\scriptsize},
  lineskip=-0.9ex,
  keywordstyle={\small\bfseries\color{SkyBlue}},  
  commentstyle={\color{OliveGreen}}, 
  stringstyle={\small\ttfamily\color{Orenge}}    
}

\begin{document}

\title{2016年度 大問1}
\author{hari64boli64 (hari64boli64@gmail.com)}
\date{\today}
\maketitle

\section{問題}

正定値対称行列の$\tr$について

\begin{align*}
  Q    & =\sqrt{B^TB}^{-1}B^T=B^T \sqrt{B^TB}^{-1} \\
  g(L) & =\tr\{(I-L)G(I-L)^T\}                     \\
\end{align*}

\section{解答}

\subsection*{(1)}

既出 (2010年大問1(1))

\subsection*{(2)}

$B^TB$は$n$次の正定値対称行列なので、$B^TB=A=R^2$と置くと、

\begin{align*}
  QB=\sqrt{B^TB}^{-1}B^TB=R^{-1}R^2=R
\end{align*}

となる。直交行列の積は直交行列になることに注意すると、$\tr(Q'R)$を最大にする直交行列$Q'$を求める問題に、本問は帰着される。もし$Q'=I$であるならば、この$Q$が確かに$\tr(QB)=\tr(R)$を最大化している。

ここで、

\begin{align*}
  \tr(Q'R) & =\tr(Q'P^TDP) \quad (R=P^TDP)      \\
           & =\tr(PQ'P^TD)                      \\
           & =\sum_{i=1}^{n}(PQ'P^T)_{ii}D_{ii} \\
\end{align*}

である。ただし、$P$は直交行列、$D$は対角行列である。これは、$R$が正定値対称行列であることから従う分解である。

直交行列の対角成分は、必ず1以下。気付くのが難しいが、証明は容易で、各行ベクトルのノルムが定義より1になることから従う。

$D$の対角成分も$R$の正定値性より正であるため、この式は$(PQ'P^T)_{ii}=1$で最大となる。

そして、そのような直交行列として$Q'=I$を取れば、その上界は達成される。よって、題意は示された。

\subsection*{(3)}

\begin{align*}
  g(L) & =\tr\{(I-L)G(I-L)^T\}   \\
       & =\tr\{G-GL^T-LG+LGL^T\} \\
       & =\tr(G+H)-\tr(GL^T+LG)  \\
       & =\tr(G+H)-2\tr(LG)      \\
\end{align*}

よって、$g(L)$の最大化は、$\tr(LG)$の最小化と等価。

(以下不明。答えはどうせ$\sqrt{H}\sqrt{G}^{-1}$だと思う)

\section{知識}

todo

\begin{itembox}[l]{問題}
  $\tr(ABC)=\tr(CAB)=\tr(BCA)$は成立しますが、
  $\tr(ABC)=\tr(ACB)$などは成立するとは限りません。
  \url{https://manabitimes.jp/math/1135}
\end{itembox}

\begin{thebibliography}{9}

\end{thebibliography}

\end{document}
