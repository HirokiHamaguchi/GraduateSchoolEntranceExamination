\documentclass[a4paper, 10pt, dvipdfmx]{jlreq}

\usepackage{amsmath,amsfonts,amssymb}
\usepackage{bm}
\usepackage{mathtools}
\usepackage{siunitx}
\usepackage[dvipdfmx]{graphicx}
\usepackage[dvipdfmx]{color}
\usepackage[dvipdfmx, colorlinks=true, allcolors=blue]{hyperref}
\usepackage{listings, jlisting}
\usepackage{tikz}
\usepackage{physics}
\usepackage{url}

\Urlmuskip=0mu plus 10mu
\allowdisplaybreaks[4]
\frenchspacing
\definecolor{OliveGreen}{rgb}{0.0,0.6,0.0}
\definecolor{Orenge}{rgb}{0.89,0.55,0}
\definecolor{SkyBlue}{rgb}{0.28, 0.28, 0.95}
\lstset{
  language={c++},
  basicstyle={\ttfamily},
  identifierstyle={\small},
  ndkeywordstyle={\small},
  frame=single,
  breaklines=true,
  numbers=left,
  xrightmargin=0zw,
  xleftmargin=3zw,
  numberstyle={\scriptsize},
  lineskip=-0.9ex,
  keywordstyle={\small\bfseries\color{SkyBlue}},  
  commentstyle={\color{OliveGreen}}, 
  stringstyle={\small\ttfamily\color{Orenge}}    
}

\begin{document}

\title{2019年度 大問1}
\author{hari64boli64 (hari64boli64@gmail.com)}
\date{\today}
\maketitle


\section{問題}

\begin{align*}
  a_{ij} \leq 0 (i,j = 1,2,\cdots,n), \quad \sum_{i=1}^{n}a_{ij}=1 (j=1,2,\cdots,n),  \quad B=\alpha A+\frac{1-\alpha}{n}\bm{1}\bm{1}^T (0<\alpha<1)
\end{align*}

\section{解答}

\subsection*{(1)}

一般に、転置行列の固有値は、固有方程式が同じになることから、元の行列と同じ固有値を取る。

$A^T$は、$\bm{1}$を固有ベクトルとして、1を固有値に持つ。

\begin{align*}
              & (A^T\bm{1})_i=\sum_{j}^n A^T_{ij}\bm{1}_j=\sum_{j}^na_{ji}=1 \\
  \Rightarrow & A^T\bm{1}=1\bm{1}
\end{align*}

そして、1より絶対値が大きな固有値を持つことはない。

固有ベクトル$\bm{x}$の、絶対値に関する最大値を$x_i$で取るとすると、

\begin{align*}
                  & (A^T\bm{x})_i =\lambda x_i        \\
  \Leftrightarrow & \sum_{j}^n a_{ji}x_j =\lambda x_i \\
\end{align*}

一方、

\begin{align*}
  \left|\sum_{j}^n a_{ji}x_j\right|\leq \sum_{j}^n |a_{ji}||x_j| \leq |x_i|\sum_{j}^n a_{ji} = |x_i| < |\lambda||x_i|=|\lambda x_i|
\end{align*}

つまり、$\lambda$が1より大きな値を取ると、この不等式に反し矛盾。

よって、$A$の固有値の絶対値で最大となるものは1である。

\subsection*{(2)}

関数$f:S \to T$を$f(x)=Bx$で定義する。

ただし、$S=\{x\in \mathbb{R}_{\geq 0}^n | \bm{1}^T\bm{x}=1 \}$である。

ここで、$S=T$を証明する。

\begin{align*}
  Bx=\alpha Ax+\frac{1-\alpha}{n}\bm{1}\bm{1}^Tx
\end{align*}
という表式と、$a_{ij},\alpha$の値域より、$T \subset \mathbb{R}_{\geq 0}^n$は明らか。

$\bm{1}^TBx=1$を示す。

\begin{align*}
  \bm{1}^TBx & =\bm{1}^T\left(\alpha Ax+\frac{1-\alpha}{n}\bm{1}\bm{1}^Tx\right) \\
             & =\alpha (\bm{1}^TA)x+\frac{1-\alpha}{n}\bm{1}^T\bm{1}\bm{1}^Tx    \\
             & =\alpha \bm{1}^Tx+\frac{1-\alpha}{n}\bm{1}^Tx                     \\
             & =\alpha+\frac{1-\alpha}{n}                                        \\
             & =1
\end{align*}

よって、$T=S$である。

$S$が$\mathbb{R}^n$の非空なコンパクト集合であることは、$S$が$\mathbb{R}^n$の有界閉集合であることから、これはコンパクト。

以上より、ヒントで与えられているブラウワーの不動点定理より、

\begin{align*}
  x\in \mathbb{R}^n,
  Bx=x, \bm{1}^Tx=1
\end{align*}

の条件を満たすような$x$が存在する。

\subsection*{(3)}

\begin{align*}
       & \left|\sum_{j=1}^nb_{ij}q_j\right|                                            \\
  =    & |(B\bm{q})_i|                                                                 \\
  =    & \left|\left((\alpha A+\frac{1-\alpha}{n}\bm{1}\bm{1}^T)\bm{q}\right)_i\right| \\
  =    & \left|(\alpha Aq)_i\right| \quad (\because \bm{1}^Tq=0)                       \\
  =    & \left|\sum_{j=1}^n\alpha a_{ij}q_j\right|                                     \\
  \leq & \sum_{j=1}^n(\alpha a_{ij})|q_j|                                              \\
  =    & \sum_{j=1}^n \left(B-\frac{1-\alpha}{n}\bm{1}\bm{1}^T\right)_{ij}|q_j|        \\
  =    & \sum_{j=1}^n \left(b_{ij}-\frac{1-\alpha}{n}\right)|q_j|                      \\
  =    & \sum_{j=1}^n b_{ij}|q_j|-\frac{1-\alpha}{n}||\bm{q}||_1                       \\
\end{align*}

\subsection*{(4)}

まず、$\bm{q}=\frac{\bm{1}}{n}-\bm{x}$と定義されるを用いると、題意は、

\begin{align*}
                  & ||B^N\frac{\bm{1}}{n}-\bm{x}||_1 \leq \alpha^N ||\frac{\bm{1}}{n}-\bm{x}||_1 \\
  \Leftrightarrow & ||B^N\frac{\bm{1}}{n}-B^N\bm{x}||_1 \leq \alpha^N ||\bm{q}||_1               \\
  \Leftrightarrow & ||B^N\bm{q}||_1 \leq \alpha^N ||\bm{q}||_1                                   \\
\end{align*}

となる。

特に、$\bm{1}^T\bm{q}=\bm{1}^T\frac{\bm{1}}{n}-\bm{1}^Tx=\frac{n}{n}-1=0$であるが、この性質は、以下に示すように、$B$を乗じても変わらない。

\begin{align*}
  \bm{1}^T(B\bm{q}) & =\bm{1}^T \left(\alpha A+\frac{1-\alpha}{n}\bm{1}\bm{1}^T\right)\bm{q} \\
                    & =\alpha \bm{1}^TA\bm{q}+\frac{1-\alpha}{n}\bm{1}^T\bm{1}\bm{1}^T\bm{q} \\
                    & =\alpha \bm{1}^T\bm{q}+\frac{1-\alpha}{n}\bm{1}^T\bm{q}                \\
                    & =\left(\alpha+\frac{1-\alpha}{n}\right)\bm{1}^T\bm{q}                  \\
                    & =0 \quad (\because \bm{1}^T\bm{q}=0)
\end{align*}

よって、$\bm{1}^T\bm{q}=0$を満たす$\bm{q}$に関して、$||B\bm{q}||_1\leq \alpha||\bm{q}||_1$であることを示せばよい。

これは、(3)で示したことから、

\begin{align*}
  |(B\bm{q})_i| & \leq \sum_{j=1}^n(\alpha a_{ij})|q_j| \\
                & \leq \alpha \sum_{j=1}^n |q_j|        \\
                & =\alpha ||\bm{q}||_1
\end{align*}

となり、直ちに従う。

\section{知識}

\subsection{コンパクト集合}

$\mathbb{R}^n$において、有界閉集合がコンパクト集合であって、閉集合がコンパクト集合である訳では無いことに注意。

例として、$\mathbb{R} \setminus (-1,1)$は閉集合であるが、当然コンパクト集合ではない。

\subsection{ブラウワーの不動点定理}

(2)のヒントは、ブラウワーの不動点定理と呼ばれている。

\section{参考文献}
\begin{thebibliography}{9}
  \bibitem{site:1}
  Mathpedia."位相空間論9:コンパクト性".2021年3月31日.\url{https://math.jp/wiki/%E4%BD%8D%E7%9B%B8%E7%A9%BA%E9%96%93%E8%AB%969%EF%BC%9A%E3%82%B3%E3%83%B3%E3%83%91%E3%82%AF%E3%83%88%E6%80%A7#.E5.AE.9A.E7.90.86_9.20_.28.24.5Cmathbb.7BR.7D.5En.24_.E3.81.AE.E3.82.B3.E3.83.B3.E3.83.91.E3.82.AF.E3.83.88.E9.9B.86.E5.90.88.29}
  \bibitem{site:2}
  Wikipedia.“ブラウワーの不動点定理”.2023年3月11日.\url{https://ja.wikipedia.org/wiki/%E3%83%96%E3%83%A9%E3%82%A6%E3%83%AF%E3%83%BC%E3%81%AE%E4%B8%8D%E5%8B%95%E7%82%B9%E5%AE%9A%E7%90%86}
\end{thebibliography}
\end{document}
