\documentclass[a4paper, 10pt, dvipdfmx]{jlreq}

\usepackage{ascmac}
\usepackage{amsmath,amsfonts,amssymb}
\usepackage{bm}
\usepackage{mathtools}
\usepackage{siunitx}
\usepackage[dvipdfmx]{graphicx}
\usepackage[dvipdfmx]{color}
\usepackage[dvipdfmx, colorlinks=true, allcolors=blue]{hyperref}
\usepackage{listings}
\usepackage{tikz}
\usepackage{physics}
\usepackage{url}

\Urlmuskip=0mu plus 10mu
\allowdisplaybreaks[4]
\frenchspacing
\definecolor{OliveGreen}{rgb}{0.0,0.6,0.0}
\definecolor{Orenge}{rgb}{0.89,0.55,0}
\definecolor{SkyBlue}{rgb}{0.28, 0.28, 0.95}
\lstset{
  language={c++},
  basicstyle={\ttfamily},
  identifierstyle={\small},
  ndkeywordstyle={\small},
  frame=single,
  breaklines=true,
  numbers=left,
  xrightmargin=0zw,
  xleftmargin=3zw,
  numberstyle={\scriptsize},
  lineskip=-0.9ex,
  keywordstyle={\small\bfseries\color{SkyBlue}},  
  commentstyle={\color{OliveGreen}}, 
  stringstyle={\small\ttfamily\color{Orenge}}    
}

\begin{document}

\title{2010年度 大問1}
\author{hari64boli64 (hari64boli64@gmail.com)}
\date{\today}
\maketitle

\section{問題}

一意な行列の分解を求めよ 難問だと思う

\section{解答}

\subsection*{(1)}

Aは正定値対称行列なので、$A=Q^TDQ$と書ける。
ただし、$Q$は直交行列、$D$は要素が全て正の対角行列である。

$D$はその平方根$\sqrt{D}$が取れることに注意すると、

\begin{align*}
  (Q^T\sqrt{D}Q)^2 & =(Q^T\sqrt{D}Q)(Q^T\sqrt{D}Q) \\
                   & =Q^TDQ                        \\
                   & =A
\end{align*}

となり、$B=Q^T\sqrt{D}Q$である。

一意性を示す。$B^2=C^2=A$とする。

$||x||_2=1$を満たす全ての$x$に対して、$B,C$の正定値性に注意すると、

\begin{align*}
  x^TBx & =\sqrt{(x^TBx)^2} \\
        & =\sqrt{x^TB^2x}   \\
        & =\sqrt{x^TC^2x}   \\
        & =\sqrt{(x^TCx)^2} \\
        & =x^TCx
\end{align*}

しかし、$||x||_2=1$を満たす全ての$x$に対して、

\begin{align*}
  x^T(B-C)x =0
\end{align*}

であると、これは$B-C=O \Leftrightarrow B=C$を意味する。

(ここの部分の証明は、対称性などを使えば決して難しい訳ではない。しかし、簡潔な証明はよく分からなかった)

よって、$B$は一意である。

\subsection*{(2)}

特異値分解より、$F=R' \Sigma S$と分解できる。
ただし、$R',S$は直交行列、$\Sigma$は全ての要素が正の対角行列である。$F$が正則な為、フルランクであり、$\Sigma$の対角成分に0はない。

よって、$F=R' (SS^T) \Sigma S=(R'S)(S^T\Sigma S) $と書けて、$R' S^T$は$R'$と$S$がそれぞれ直交行列な為、直交行列である。

以上より、$R=R' S^T,U=S^T \Sigma S$とすれば良い。

一意性については、$F^TF=U^TR^TRU=U^TU=U^2=W$とすると、$W$は正定値対称行列だが、(1)より、$U$は一意と分かる。よって、$R$も一意である。

全体で一意。

\subsection*{(3)}

\begin{align*}
  FR^T & =RUR^T       \\
       & =RQ'D'Q^TR^T
\end{align*}

より、$V=FR^T$は正定値対称行列。よって、$F=VR$となる。一意性は自明。

\section{知識}

2020年度問1をまず参照のこと。

\subsection*{(1)について}

計数の人のブログらしきものがあった。\cite{site:2}

\subsection*{(2)}

直交行列と直交行列の積は直交行列であることは、簡単に示せる。

\begin{align*}
  (RS)^T(RS)=S^TR^TRS=I
\end{align*}

特異値分解についてまとめる。

\begin{itembox}[l]{特異値分解の定義,性質,具体例\cite{site:3}}
  特異値分解とは、$m\times n$行列$A$を$A=U \Sigma V$と分解することです。ただし、

  \begin{quote}
    \begin{itemize}
      \item $U$は$m \times m$の直交行列(各列が互いに直交する行列)
      \item $V$は$n \times n$の直交行列
      \item $\Sigma$は ... 非対角成分は0、対角成分は非負で大きさの順に並んだ行列
    \end{itemize}
  \end{quote}

  です。任意の行列はこのように分解できます。また,$Σ$の対角成分で 0でないもの(0を含めることもある)を特異値と言います。
\end{itembox}

なお、行列の(0でない)特異値の数は、その行列のランクと一致する。

以下に、参考文献の続きを載せると、

\begin{itembox}[l]{特異値分解の定義,性質,具体例\cite{site:3}}
  $A$が対称行列のとき、$A$の固有値と特異値は一致します。対称行列は直交行列で対角化できるからです。

  $A^TA$の0でない固有値の正の平方根は$A$の特異値です。理由は、$A=U\Sigma V$と特異値分解できるとき$A^TA=V^T(\Sigma^T\Sigma)V$と固有値分解できるからです。同様に$AA^T$の0でない固有値の正の平方根も$A$の特異値です。
\end{itembox}

となっている。

\subsection*{行列分解}

本問を解く際に、特異値分解以外の分解に関しても考慮したが、知識が足りず苦労した。それを以下に簡単にまとめる。

QR分解を用いて解くことも出来るのだろうか?

\subsubsection*{LU分解}

以下、参考文献\cite{site:4}より引用。

正方行列$A$を下三角行列$L$と上三角行列$U$との積によって、$A=LU$と表すことをLU分解(lower-upper(LU) decomposition)という。

行列がLU分解可能であるための必要十分条件は、全ての主座小行列の行列式が0でないことである。

\subsection*{QR分解}

以下、参考文献\cite{site:5}より引用。

QR分解(キューアールぶんかい、英: QR decomposition, QR factorization)とは、$m \times n$ 実行列$A$を、$m$次直交行列$Q$と$m \times  n$上三角行列$R$との積への分解により表すこと、またはそう表した表現をいう。このような分解は常に存在する。

QR分解は線型最小二乗問題を解くために使用される。また、固有値問題の数値解法の1つであるQR法の基礎となっている。

すべての実正方行列$A$は直交行列$Q$と上三角行列(別名右三角行列)$R$を用いて$A=QR$と分解できる。もし$A$が正則ならば、$R$の対角成分が正になるような因数分解は一意に定まる。

QR分解を計算する手法として、グラム・シュミット分解、ハウスホルダー変換、ギブンス回転などがある。

\subsection*{コレスキー分解}

以下、参考文献\cite{site:6}より引用。

行列$A$を下三角行列$C$とその転置行列の $C^T$の積に分解すること、すなわち、$A=CC^T$と分解することを、コレスキー分解 (cholesky decomposition) という。

行列$A$が正定値行列であるならば、 コレスキー分解可能である。

\begin{thebibliography}{9}
  \bibitem{site:1}
  高校数学の美しい物語."同時対角化可能⇔交換可能の意味と証明".2021年3月7日.\url{https://manabitimes.jp/math/1196}
  \bibitem{site:2}
  Lilliput Steps.“半正定値行列の平方根行列の存在と一意性”.HatenaBlog.2020年5月6日.\url{https://kagamiz.hatenablog.com/entry/2020/05/16/212214}
  \bibitem{site:3}
  高校数学の美しい物語."特異値分解の定義,性質,具体例".2022年11月3日.\url{https://manabitimes.jp/math/1280}
  \bibitem{site:4}
  理数アラカルト."LU分解を解説 ~具体例と必要十分条件~".2022年4月17日.\url{http://risalc.info/src/LU-decomposition.html}
  \bibitem{site:5}
  Wikipedia."QR分解".2022年11月20日.\url{https://ja.wikipedia.org/wiki/QR%E5%88%86%E8%A7%A3}
  \bibitem{site:6}
  理数アラカルト."コレスキー分解とは?".2023年4月2日.\url{http://risalc.info/src/cholesky-decomposition.html}
\end{thebibliography}

\end{document}
