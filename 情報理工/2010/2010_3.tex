\documentclass[a4paper, 10pt, dvipdfmx]{jlreq}

\usepackage{amsmath,amsfonts,amssymb}
\usepackage{bm}
\usepackage{mathtools}
\usepackage{siunitx}
\usepackage[dvipdfmx]{graphicx}
\usepackage[dvipdfmx]{color}
\usepackage[dvipdfmx, colorlinks=true, allcolors=blue]{hyperref}
\usepackage{listings, jlisting}
\usepackage{tikz}
\usepackage{physics}
\usepackage{url}

\Urlmuskip=0mu plus 10mu
\allowdisplaybreaks[4]
\frenchspacing
\definecolor{OliveGreen}{rgb}{0.0,0.6,0.0}
\definecolor{Orenge}{rgb}{0.89,0.55,0}
\definecolor{SkyBlue}{rgb}{0.28, 0.28, 0.95}
\lstset{
  language={c++},
  basicstyle={\ttfamily},
  identifierstyle={\small},
  ndkeywordstyle={\small},
  frame=single,
  breaklines=true,
  numbers=left,
  xrightmargin=0zw,
  xleftmargin=3zw,
  numberstyle={\scriptsize},
  lineskip=-0.9ex,
  keywordstyle={\small\bfseries\color{SkyBlue}},  
  commentstyle={\color{OliveGreen}}, 
  stringstyle={\small\ttfamily\color{Orenge}}    
}

\begin{document}

\title{2010年度 大問3}
\author{hari64boli64 (hari64boli64@gmail.com)}
\date{\today}
\maketitle

\section{問題}

$n$次元ユークリッド空間の有界集合について

\section{解答}

\subsection*{(1)}

$C$の範囲内に移動させる方法が、かならず1通りのみである。

これを厳密に書くのは、測度論などの都合上、かなり難しい気がする。

……と思っていたが、よくよく考えると自明かも知れない。

\subsection*{(2)}

そうでなければ、体積が1より大きいということに矛盾する。

\subsection*{(3)}

(2)から自明。

\subsection*{(4)}

$v(\frac{1}{2}B)=\frac{1}{2^n}v(B) > 1$より、(3)の結果から、

\begin{align*}
                  & \exists \bm{x},\bm{y} \in \frac{1}{2}B \text{ s.t. } \bm{x}-\bm{y} \in \mathbb{Z}^n \\
  \Leftrightarrow & \exists \bm{2x},\bm{2y} \in B \text{ s.t. } \bm{x}-\bm{y} \in \mathbb{Z}^n          \\
\end{align*}

$B$の凸性と原点対称性から、$\frac{1}{2}\qty(\bm{2x}+(-\bm{2y}))=\bm{x}-\bm{y}$も$B$に含まれる。

特に、$\bm{x} \neq \bm{y}$より、$\bm{x}-\bm{y} \neq \bm{0}$である。

以上より、$\bm{x}-\bm{y}$が題意を満たす。

\subsection*{(5)}

\begin{align*}
  B & =\left\{\{g_j\}_{1 \leq j\leq 3} \middle| \sum_{i=1}^{3} \left\lvert \sum_{j=1}^{3}r_{ij}g_j \right\rvert < \alpha \right\} \\
    & =\left\{\bm{g}\in \mathbb{R}^3 \middle| ||R\bm{g}||_1 < \alpha \right\}
\end{align*}

Bが原点対称な有界凸集合であることは明らか。

あとは、$v(B)>2^n=2^3$を示せば、(4)より従う。

行列式の6倍は四面体の体積を表すことに注意すると、

\begin{align*}
  v(B)=2^3 \alpha^3 \frac{1}{6\det R} \leq 2^3
\end{align*}

(ここ、もう少し説明のしようがあると思われるが、あまり良い説明が思いつかない)

よって、示された。

\section{知識}

行列式の6倍は四面体の体積を表す。\cite{site:1}

\begin{thebibliography}{9}
  \bibitem{site:1}
  高校数学の美しい物語."四面体の体積を求める2つの公式with行列式".2021年3月7日.\url{https://manabitimes.jp/math/1012}
\end{thebibliography}

\end{document}
