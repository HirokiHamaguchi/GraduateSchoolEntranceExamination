\documentclass[a4paper, 10pt, dvipdfmx]{jlreq}

\usepackage{amsmath,amsfonts,amssymb}
\usepackage{bm}
\usepackage{mathtools}
\usepackage{siunitx}
\usepackage[dvipdfmx]{graphicx}
\usepackage[dvipdfmx]{color}
\usepackage[dvipdfmx, colorlinks=true, allcolors=blue]{hyperref}
\usepackage{listings, jlisting}
\usepackage{tikz}
\usepackage{physics}
\usepackage{url}

\Urlmuskip=0mu plus 10mu
\allowdisplaybreaks[4]
\frenchspacing
\definecolor{OliveGreen}{rgb}{0.0,0.6,0.0}
\definecolor{Orenge}{rgb}{0.89,0.55,0}
\definecolor{SkyBlue}{rgb}{0.28, 0.28, 0.95}
\lstset{
  language={c++},
  basicstyle={\ttfamily},
  identifierstyle={\small},
  ndkeywordstyle={\small},
  frame=single,
  breaklines=true,
  numbers=left,
  xrightmargin=0zw,
  xleftmargin=3zw,
  numberstyle={\scriptsize},
  lineskip=-0.9ex,
  keywordstyle={\small\bfseries\color{SkyBlue}},  
  commentstyle={\color{OliveGreen}}, 
  stringstyle={\small\ttfamily\color{Orenge}}    
}

\begin{document}

\title{2015年度 大問1}
\author{hari64boli64 (hari64boli64@gmail.com)}
\date{\today}
\maketitle

\section{問題}

\begin{align*}
  \text{maximize}_{\bm{u},\bm{v}} \quad & \bm{a}^T(\bm{u}-\bm{v})              \\
  \text{subject to} \quad               & \bm{1}^T\bm{u}+\bm{1}^T\bm{v} \leq 1 \\
                                        & \bm{u}\geq \bm{0},\bm{v}\geq \bm{0}  \\
\end{align*}


\begin{align*}
  \text{minimize}_{\bm{x}} \quad & \bm{x}^TQ^{-1}\bm{x}    \\
  \text{subject to} \quad        & ||\bm{x}||_{\infty} = 1 \\
\end{align*}

\section{解答}

\subsection*{(1)}

\subsubsection*{(1-1)}

自明

\subsubsection*{(1-2)}

todo

\subsubsection*{(1-3)}

自明

\subsection*{(2)}

\subsubsection*{(2-1)}

まず、$||\bm{x}||_2=1$という制約に変更した問題を解く。

$Q=P^TDP$と分解する。ただし、$P$は直交行列で、$D$は対角行列であり、

\begin{align*}
  D=\mqty(d_1 &  & \\ & \ddots & \\ & & d_n)
\end{align*}

とする。

$Q^{-1}=P^TD^{-1}P$である。なお、正定値性より、$D^{-1}$は定義可能である。

その上で、

\begin{align*}
  \bm{x}^TQ^{-1}\bm{x} & = \bm{x}^TP^TD^{-1}P\bm{x}                                                      \\
                       & = (P\bm{x})^TD^{-1}(P\bm{x})                                                    \\
                       & = \bm{y}^TD^{-1}\bm{y} \quad (\bm{y}=P\bm{x},||\bm{y}||_2=\bm{x}^TP^TP\bm{x}=1) \\
                       & = \sum_{i=1}^{n}\frac{y_i^2}{d_i}                                               \\
\end{align*}

となるので、この制約下における解は、$Q$の最大固有値を$d_{\text{max}}$として、$\frac{1}{d_{\text{max}}}$となる。

???

\subsubsection*{(2-2)}

todo


\end{document}
