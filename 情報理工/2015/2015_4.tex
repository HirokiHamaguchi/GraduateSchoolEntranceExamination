\documentclass[a4paper, 10pt, dvipdfmx]{jlreq}

\usepackage{amsmath,amsfonts,amssymb}
\usepackage{bm}
\usepackage{mathtools}
\usepackage{siunitx}
\usepackage[dvipdfmx]{graphicx}
\usepackage[dvipdfmx]{color}
\usepackage[dvipdfmx, colorlinks=true, allcolors=blue]{hyperref}
\usepackage{listings, jlisting}
\usepackage{tikz}
\usepackage{physics}
\usepackage{url}

\Urlmuskip=0mu plus 10mu
\allowdisplaybreaks[4]
\frenchspacing
\definecolor{OliveGreen}{rgb}{0.0,0.6,0.0}
\definecolor{Orenge}{rgb}{0.89,0.55,0}
\definecolor{SkyBlue}{rgb}{0.28, 0.28, 0.95}
\lstset{
  language={c++},
  basicstyle={\ttfamily},
  identifierstyle={\small},
  ndkeywordstyle={\small},
  frame=single,
  breaklines=true,
  numbers=left,
  xrightmargin=0zw,
  xleftmargin=3zw,
  numberstyle={\scriptsize},
  lineskip=-0.9ex,
  keywordstyle={\small\bfseries\color{SkyBlue}},  
  commentstyle={\color{OliveGreen}}, 
  stringstyle={\small\ttfamily\color{Orenge}}    
}

\begin{document}

\title{2015年度 大問4}
\author{hari64boli64 (hari64boli64@gmail.com)}
\date{\today}
\maketitle

\section{問題}

\section{解答}

\subsection*{(1)}

\subsection*{(2)}

\subsection*{(3)}

\section{知識}


\section{おまけ}

\lstinputlisting[caption=calculateLimit,label=code:calculateLimit,language=Python]{4.py}

\begin{lstlisting}[caption=result, label=code:result]
n: 1, A^n: [[0, 1], [1, 1]]
trace: 1
n: 2, A^n: [[1, 1], [1, 2]]
trace: 3
n: 3, A^n: [[1, 2], [2, 3]]
trace: 4
n: 4, A^n: [[2, 3], [3, 5]]
trace: 7
n: 5, A^n: [[3, 5], [5, 8]]
trace: 11
n: 6, A^n: [[5, 8], [8, 13]]
trace: 18
n: 7, A^n: [[8, 13], [13, 21]]
trace: 29
n: 8, A^n: [[13, 21], [21, 34]]
trace: 47
n: 9, A^n: [[21, 34], [34, 55]]
trace: 76
==========
limit of logN(p)/p:  0.4812118250596034
math.log((1+math.sqrt(5))/2)=0.48121182505960347
\end{lstlisting}

\begin{thebibliography}{9}

\end{thebibliography}

\end{document}
