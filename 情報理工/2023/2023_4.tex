\documentclass[a4paper, 10pt, dvipdfmx]{jlreq}

\usepackage{amsmath,amsfonts,amssymb}
\usepackage{bm}
\usepackage{mathtools}
\usepackage{siunitx}
\usepackage[dvipdfmx]{graphicx}
\usepackage[dvipdfmx]{color}
\usepackage[dvipdfmx, colorlinks=true, allcolors=blue]{hyperref}
\usepackage{listings}
\usepackage{tikz}
\usepackage{physics}
\usepackage{url}

\Urlmuskip=0mu plus 10mu
\allowdisplaybreaks[4]
\frenchspacing
\definecolor{OliveGreen}{rgb}{0.0,0.6,0.0}
\definecolor{Orenge}{rgb}{0.89,0.55,0}
\definecolor{SkyBlue}{rgb}{0.28, 0.28, 0.95}
\lstset{
  language={c++},
  basicstyle={\ttfamily},
  identifierstyle={\small},
  ndkeywordstyle={\small},
  frame=single,
  breaklines=true,
  numbers=left,
  xrightmargin=0zw,
  xleftmargin=3zw,
  numberstyle={\scriptsize},
  lineskip=-0.9ex,
  keywordstyle={\small\bfseries\color{SkyBlue}},  
  commentstyle={\color{OliveGreen}}, 
  stringstyle={\small\ttfamily\color{Orenge}}    
}

\begin{document}

\title{2023年度 大問4}
\author{hari64boli64 (hari64boli64@gmail.com)}
\date{\today}
\maketitle

\section{問題}

$Y=\frac{1}{X^2}$

\section{解答}

\subsection*{(1)}

確率変数の変数変換の公式を用いると、
また、$X=\pm\frac{1}{\sqrt{Y}}$と、2つの解があることに注意すると、
\begin{align*}
  f(y) & = \begin{cases}
             2 \frac{1}{\sqrt{2\pi}}e^{-\frac{1}{2y}} \abs{\dv{x}{y}} & (0<y)              \\
             0                                                        & (\text{otherwise})
           \end{cases} \\
       & = \begin{cases}
             \frac{1}{\sqrt{2\pi}}e^{-\frac{1}{2y}} y^{-\frac{3}{2}} & (0<y)              \\
             0                                                       & (\text{otherwise})
           \end{cases}
\end{align*}
となる。

なお、実際、$\int_0^{\infty}\frac{1}{\sqrt{2\pi}}e^{-\frac{1}{2y}} y^{-\frac{3}{2}}\dd{y}=1$である。
(Wolfram Alphaで計算)

\subsection*{(2)}

\begin{align*}
  \dv{L(u)}{u} & =\dv{u} \int_0^{\infty} {e^{-uy}f(y)}\dd{y}                                                                        \\
               & =\int_0^{\infty} \dv{u} e^{-uy}f(y) \dd{y}                                                                         \\
               & =\int_0^{\infty}  {-y e^{-uy}f(y)} \dd{y}                                                                          \\
               & =-\frac{1}{\sqrt{2\pi}}\int_0^{\infty}  {e^{-uy}e^{-\frac{1}{2y}}y^{-\frac{1}{2}}} \dd{y}                          \\
               & =\frac{1}{2u\sqrt{2\pi}} \int_0^{\infty} {e^{-\frac{1}{2z}}e^{-uz}\qty(\frac{1}{2uz})^{-\frac{1}{2}}z^{-2}} \dd{z} \\
               & =\sqrt{2u}\frac{1}{\sqrt{2\pi}}\int_0^\infty e^{-\frac{1}{2z}}e^{-uz}z^{-\frac{3}{2}} \dd{z}                       \\
               & =\sqrt{2u}L(u)
\end{align*}

微分と積分が交換できることは、$f(y)$が確率密度関数であるため被積分関数が可積分であり、また、微分後の式も可積分であることから従う。

$\frac{1}{2y}=uz$という変数変換が本質的。
見つけた人曰く、「$\exp(-uy),\exp(-\frac{1}{2}y)$の形を保存するにはどうすればいいのかなぁと思って、いろいろ気合いで推測」したらしいです。

\subsection*{(3)}

(2)の式に当てはめて$L(-iu)=e^{-\sqrt{2u}\sqrt{-i}}$としたい所だが、これは厳密な解答ではない。

以下Slackにあがっていた解答の書き起こし。
なお、この解答の作成者は数学科の複素関数論を履修されていた方です。
(私には無理……)

\begin{align*}
  D & =\left\{ re^{i\theta}\in \mathbb{C} | r>0, -\frac{\pi}{2} < \theta < \frac{\pi}{2} \right\} \\
  E & =\left\{ re^{i\theta}\in \mathbb{C} | r>0, -\frac{\pi}{4} < \theta < \frac{\pi}{4} \right\}
\end{align*}
とおく。

\begin{align*}
  E \to D, \; & z \mapsto z^2                                                                  \\
  D \to E, \; & z \mapsto \sqrt{z} \; \qty(re^{i\theta} \mapsto \sqrt{r}e^{i\frac{\theta}{2}})
\end{align*}
は正則かつ全単射で互いに逆写像。

ここで、$L$は$D$で正則、$\overline{D}$で連続である。この証明は後述。

実軸正の部分で、
\begin{equation*}
  \dv{L}{u} {}(u)=-\frac{1}{\sqrt{2u}}L(u)
\end{equation*}
が成り立つから、両辺が$D$上で正則な為、一致の定理より、$D$上
\begin{equation*}
  \dv{L}{z} {}(z)=-\frac{1}{\sqrt{2z}}L(z)
\end{equation*}
が成立する。

$E$上で$M(z)=L(z^2)$とおけば、上式より、
\begin{align*}
  \dv{M}{z}(z) & =\dv{L}{z}{}(z^2)2z                                          \\
               & =-\frac{1}{\sqrt{2z^2}}L(z^2)2z                              \\
               & = -\sqrt{2}M(z) \; (\because \sqrt{z^2}=z \: \text{on} \: E)
\end{align*}

これより、$E$上$M(z)=Ce^{-\sqrt{2}z}$となる。ただし、$C$は積分定数。
実軸正の部分から、$u\searrow0$とすれば、$M(u)=L(\sqrt{u}) \to L(0)=1$となる。
従って、$C=1$で$M(z)=e^{-\sqrt{2}z}$、$D$上で$L(z)=M(\sqrt{z})=e^{-\sqrt{2z}}$となる。

$u>0$の場合、$z=a-iu$として、$a\searrow0$とすれば、
$\sqrt{z} \to \sqrt{u} \qty(\frac{1}{\sqrt{2}}-\frac{1}{\sqrt{2}}i)$なので、
\begin{equation*}
  \phi(u)=L(-iu)=\lim_{a \searrow 0}L(a-iu)=e^{-\sqrt{u}(1-i)}
\end{equation*}
となる。極限を取る際に$\overline{D}$上での連続性を用いた。

$u<0$の場合、$z=a-iu$として、$a\searrow0$とすれば、
$\sqrt{z} \to \sqrt{u} \qty(\frac{1}{\sqrt{2}}+\frac{1}{\sqrt{2}}i)$なので、
\begin{equation*}
  \phi(u)=L(-iu)=\lim_{a \searrow 0}L(a-iu)=e^{-\sqrt{-u}(1+i)}
\end{equation*}
となる。極限を取る際に$\overline{D}$上での連続性を用いた。

u=0の場合、
\begin{equation*}
  \phi(0)=1
\end{equation*}
となる。

以上より、まとめて、
\begin{equation*}
  \phi(u)=\begin{cases}
    e^{-\sqrt{u}(1-i)}  & (u>0) \\
    e^{-\sqrt{-u}(1+i)} & (u<0) \\
    0                   & (u=0)
  \end{cases}
\end{equation*}
が答えとなる。

以下、$L$の正則性と連続性を見る。

まず正則性を見る。
$u=a+ib \in D$とする。
$0<\abs{h}<\frac{a}{2}$とし、$h=p+iq$とおく。

\begin{equation*}
  \frac{L(u+h)-L(u)}{h}=\int_0^\infty \frac{e^{-(u+h)y}-e^{-uy}}{h} f(y) \dd{y}
\end{equation*}
で、被積分関数に関して、
\begin{align*}
  \abs{\frac{e^{-(u+h)y}-e^{-uy}}{h}} & = \abs{\frac{1}{h}\int_u^{u+h} -y e^{-ky} \dd{k}}                  \\
                                      & = \abs{\frac{1}{h}\int_0^1 -y e^{-\qty(u+t(p+iq))y} (p+iq) \dd{t}} \\
                                      & \leq \frac{1}{\abs{h}}\int_0^1 y e^{-\qty(a+tp)y} \abs{h} \dd{t}   \\
                                      & \leq \int_0^1 y e^{-\qty(a-\frac{\abs{p}}{2})y} \dd{t}             \\
                                      & \leq y e^{-\qty(a-\frac{\abs{p}}{2})y}
\end{align*}
となるため、$y e^{-\qty(a-\frac{\abs{p}}{2})y} f(y)$は可積分であり、優収束定理が使えて、
\begin{equation*}
  \dv{L(u)}{u} =\int_0^{\infty}  {-y e^{-uy}f(y)} \dd{y}
\end{equation*}
となる。つまり、$D$上正則となる。

次に連続性を見る。
$u=ib \in \overline{D}, |h|<1,u+h \in \overline{D}$とし、
$h=p+iq (p \geq 0)$とおく。

\begin{equation*}
  L(u+h)=\int_0^{\infty} e^{-(u+h)y}f(y)\dd{y}
\end{equation*}
となるが、被積分関数に関して、
\begin{equation*}
  \abs{e^{-(u+h)y}f(y)}=e^{-py}f(y) \leq f(y) \; (\because \; y \geq 0)
\end{equation*}
なので、優収束定理が使えて、
\begin{equation*}
  \lim_{z \to u}L(z)=\int_0^\infty e^{-uy}f(y) \dd{y}
\end{equation*}
となる。つまり、$\overline{D}$上連続となる。

\subsection*{(4)}

$\frac{1}{n^2}\sum_{i=1}^n{Y_i}$の特性関数を求める。

\begin{align*}
  \mathbb{E}\qty[e^{iu\frac{1}{n^2}\sum_{i=1}^n{Y_i}}] & =\mathbb{E}\qty[\prod_{i=1}^{n}e^{iu\frac{1}{n^2}Y_i}]                             \\
                                                       & =\prod_{i=1}^{n}\mathbb{E}\qty[e^{iu\frac{1}{n^2}Y_i}] \; (\because \text{i.i.d.}) \\
                                                       & =\mathbb{E}\qty[e^{iu\frac{1}{n^2}Y}]^n                                            \\
                                                       & =\phi\qty(\frac{u}{n^2})^n                                                         \\
                                                       & =\phi(u)
\end{align*}
となる。

よって、Yに分布収束する($n \to \infty$が関係ない答えである為、本当に正しいのかどうかは不明)。

\section{知識}

確率変数の変数変換の公式\cite{label:1}

単調写像でない場合に注意\cite{label:2}

\begin{align*}
              & P(x<X<x+\delta x)=P(y<Y<y+\delta y)                              \\
  \Rightarrow & \int_{x}^{x+\dd{x}}f(x')\dd{x'}= \int_{y}^{y+\dd{y}}g(y')\dd{y'} \\
  \Rightarrow & f(x)\dd{x}=g(y)\dd{y}                                            \\
  \Rightarrow & g(y)=f(x)\abs{\dv{x}{y}}
\end{align*}

\begin{thebibliography}{9}
  \bibitem{label:1}
  石川顕一. ``統計数理''. 東京大学工学部システム創成学科Cコース. \\\url{https://ocw.u-tokyo.ac.jp/lecture_files/engin_05/2/notes/ja/ishikawa2.pdf}
  \bibitem{label:2}
  ``【確率変数の変換】証明と例題''. ゆっくりキカイガクシュウ. 2020/2/11. \\\url{https://laid-back-scientist.com/change-of-variables}
\end{thebibliography}

\end{document}
