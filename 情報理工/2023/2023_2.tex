\documentclass[a4paper, 10pt, dvipdfmx]{jlreq}

\usepackage{amsmath,amsfonts,amssymb}
\usepackage{bm}
\usepackage{mathtools}
\usepackage{siunitx}
\usepackage[dvipdfmx]{graphicx}
\usepackage[dvipdfmx]{color}
\usepackage[dvipdfmx, colorlinks=true, allcolors=blue]{hyperref}
\usepackage{listings}
\usepackage{tikz}
\usepackage{physics}
\usepackage{url}

\Urlmuskip=0mu plus 10mu
\allowdisplaybreaks[4]
\frenchspacing
\definecolor{OliveGreen}{rgb}{0.0,0.6,0.0}
\definecolor{Orenge}{rgb}{0.89,0.55,0}
\definecolor{SkyBlue}{rgb}{0.28, 0.28, 0.95}
\lstset{
  language={c++},
  basicstyle={\ttfamily},
  identifierstyle={\small},
  ndkeywordstyle={\small},
  frame=single,
  breaklines=true,
  numbers=left,
  xrightmargin=0zw,
  xleftmargin=3zw,
  numberstyle={\scriptsize},
  lineskip=-0.9ex,
  keywordstyle={\small\bfseries\color{SkyBlue}},  
  commentstyle={\color{OliveGreen}}, 
  stringstyle={\small\ttfamily\color{Orenge}}    
}

\begin{document}

\title{2023年度 大問2}
\author{hari64boli64 (hari64boli64@gmail.com)}
\date{\today}
\maketitle

\section{問題}

\begin{align*}
  F_i(x)=x_i \qty(v_i+\sum_{j=1}^{n}a_{ij}x_j) \\
  \dv{x_i(t)}{t}=F_i(x(t))
\end{align*}

\section{解答}

\subsection*{(1)}

まず、$F_i(x^*)=0$という条件より、
また、$x^* \in \mathcal{X}_n$より、$x^*_i \neq 0$であることより、
$v_i=-\sum_{j=1}^{n}{a_{ij}x^*_j}$である。

条件を整理していくと、
\begin{align*}
  \dot{L(x')} & =\left.\dv{L(x(t))}{dt}\right|_{t=0}                                                                \\
              & =\sum_{i=1}^{n}{c_i\qty(-\frac{x_i^*}{x'_i}+1)\left.\dv{x_i}{t}\right|_{t=0}}                       \\
              & =\sum_{i=1}^{n}{c_i\qty(-\frac{x_i^*}{x'_i}+1)F_i(x')}                                              \\
              & =\sum_{i=1}^{n}{c_i\qty(-\frac{x_i^*}{x'_i}+1)x'_i\qty(v_i+\sum_{j=1}^{n}a_{ij}x'_j)}               \\
              & =\sum_{i=1}^{n}{c_i\qty(-x_i^*+x'_i)\qty(v_i+\sum_{j=1}^{n}a_{ij}x'_j)}                             \\
              & =-\sum_{i=1}^{n}{c_ix_i^*v_i}-\sum_{i=1}^{n}{c_ix_i^*\qty(\sum_{j=1}^{n}a_{ij}x'_j)}
  +\sum_{i=1}^{n}{c_ix'_iv_i}+\sum_{i=1}^{n}{c_ix'_i\qty(\sum_{j=1}^{n}a_{ij}x'_j)}                                 \\
              & =\sum_{i=1}^{n}\sum_{j=1}^{n}{c_ix_i^*a_{ij}x^*_j}-\sum_{i=1}^{n}\sum_{j=1}^{n}{c_ix_i^*a_{ij}x'_j}
  -\sum_{i=1}^{n}\sum_{j=1}^{n}{c_ix'_ia_{ij}x^*_j}+\sum_{i=1}^{n}\sum_{j=1}^{n}{c_ix'_ia_{ij}x'_j}                 \\
              & ={x^*}^TCAx^*-{x^*}^TCAx'-{x'}^TCAx^*+{x'}^TCAx'                                                    \\
              & =(x^*-x')^TCA(x^*-x')
\end{align*}
となる。

よって、
\begin{align*}
                  & (\forall x'\in \mathcal{X}_n\setminus \{x^*\}) \dot{L(x')} < 0                                                      \\
  \Leftrightarrow & (\forall x'\in \mathcal{X}_n\setminus \{x^*\})(x^*-x')^TCA(x^*-x')<0                                                \\
  \Leftrightarrow & (\forall x'\in \mathcal{X}_n\setminus \{x^*\}) \qty((x^*-x')^TCA(x^*-x')< 0) \wedge \qty((x^*-x')^TA^TC(x^*-x')< 0) \\
  \Leftrightarrow & (\forall y \in \mathbb{R}^n) y^T(CA+A^TC)y< 0                                                                       \\
  \Leftrightarrow & CA+A^TC \prec 0
\end{align*}
となる。
なお、$x^*-x'$から$y$への変換には注意が必要である。
全ての$y\in \mathbb{R}^n$に対して、
$x^*-x'=y$を満たす$x'\in \mathcal{X}_n\setminus \{x^*\}$
が存在するとは限らない。
ここでは長さに関する定数倍の変換が挟まっている。

\subsection*{(2)}

まず、$\nabla H_w(z)= \qty(w_1z_1,\dots,w_nz_n)^T$である。

ここで、
\begin{align*}
  \dv{z_i(t)}{t} & =\dv{x_i(t)}{t}                                               \\
                 & =F_i(x(t))                                                    \\
                 & =x_i(t)\qty(v_i+\sum_{j=1}^{n}a_{ij}x_j(t))                   \\
                 & =(x_i(t)-x^*_i+x^*_i)\qty(\sum_{j=1}^{n}a_{ij}(x_j(t)-x^*_j)) \\
                 & =(z_i(t)+x^*_i)\qty(\sum_{j=1}^{n}a_{ij}z_j(t))
\end{align*}
となるので、
\begin{align*}
  \dv{z(t)}{t} & =(Z+X^*) A z(t)                    \\
               & =(Z+X^*) A W^{-1} W z(t)           \\
               & =(Z+X^*) A W^{-1} \nabla H_w(z(t))
\end{align*}
となる。

\subsection*{(3)}

条件をより平易な表現で言い表すと、$\lim_{z(t) \to 0}$において、
$\dv{H_w(z(t))}{t}<0$を満たすような$w$を、$c$を用いて表現せよ、
ということになる。

このことを念頭に置いて式変形していくと、
\begin{align*}
  \dv{H_w(z(t))}{t} & =\sum_{i=1}^{n}{w_iz_i(t)\dv{z_i(t)}{t}}                \\
                    & =\nabla H_w(z(t))^TG(z(t))\nabla H_w(z(t))              \\
                    & =\nabla H_w(z(t))^T\qty((Z+X^*)AW^{-1})\nabla H_w(z(t))
\end{align*}
となり、
\begin{align*}
                  & \dv{H_w(z(t))}{t}<0                                                    \\
  \Leftrightarrow & \nabla H_w(z(t))^T\qty((Z+X^*)AW^{-1})\nabla H_w(z(t))<0               \\
  \Leftrightarrow & \nabla H_w(z(t))^T\qty((Z+X^*)C^{-1}(CA+A^TC)W^{-1})\nabla H_w(z(t))<0
\end{align*}
となる。
なお、最後の変形で、$W,X^*,Z$が対角行列であることを用いた。

以上より、$(Z+X^*)C^{-1}(CA+A^TC)W^{-1}$が$CA+A^TC$に関する
二次形式の形で記述出来る時、これはその負定値性より負になる。

これは、
\begin{align*}
                  & {W^{-1}}^T=(Z+X^*)C^{-1} \\
  \Leftrightarrow & W=(Z+X^*)^{-1}C
\end{align*}
を意味し、$Z \to 0$を考慮すると、$W=X^*C$を満たすような$w$であれば、
題意を満たすことが分かる。

\end{document}
