\documentclass[a4paper, 10pt, dvipdfmx]{jlreq}

\usepackage{ascmac}
\usepackage{amsmath,amsfonts,amssymb}
\usepackage{bm}
\usepackage{mathtools}
\usepackage{siunitx}
\usepackage[dvipdfmx]{graphicx}
\usepackage[dvipdfmx]{color}
\usepackage[dvipdfmx, colorlinks=true, allcolors=blue]{hyperref}
\usepackage{listings, jlisting}
\usepackage{tikz}
\usepackage{physics}
\usepackage{url}

\Urlmuskip=0mu plus 10mu
\allowdisplaybreaks[4]
\frenchspacing
\definecolor{OliveGreen}{rgb}{0.0,0.6,0.0}
\definecolor{Orenge}{rgb}{0.89,0.55,0}
\definecolor{SkyBlue}{rgb}{0.28, 0.28, 0.95}
\lstset{
  language={c++},
  basicstyle={\ttfamily},
  identifierstyle={\small},
  ndkeywordstyle={\small},
  frame=single,
  breaklines=true,
  numbers=left,
  xrightmargin=0zw,
  xleftmargin=3zw,
  numberstyle={\scriptsize},
  lineskip=-0.9ex,
  keywordstyle={\small\bfseries\color{SkyBlue}},  
  commentstyle={\color{OliveGreen}}, 
  stringstyle={\small\ttfamily\color{Orenge}}    
}

\begin{document}

\title{2020年度 大問3}
\author{hari64boli64 (hari64boli64@gmail.com)}
\date{\today}
\maketitle


\section{問題}

\begin{align*}
  L & =K[x_1,x_2,\cdots ,x_n, x_1^{-1},x_2^{-1},\cdots ,x_n^{-1}] \\
  R & =K[x_1,x_2,\cdots ,x_n, y_1,y_2,\cdots ,y_n]
\end{align*}

\section{解答}

\subsection*{(1)}

以下の二つを言えばよい。

\begin{quote}
  \begin{itemize}
    \item  $\varphi^{-1}(J)$は加法について部分群である

          $J$も加法について$L$の部分群であるので明らか。

    \item $r \in \varphi^{-1}(J),x \in R \Rightarrow rx \in \varphi^{-1}(J)$

          $\varphi(rx)=\varphi(r)\varphi(x)\in J \quad (\because \varphi(r) \in J , \varphi(x) \in L)$より従う。
  \end{itemize}
\end{quote}


\subsection*{(2)}

自明

\subsection*{(3)}

$I \subset \mathrm{Ker}\varphi$は代入すれば明らか。

$\mathrm{Ker}\varphi \subset I$は、(2)より$r\neq 0$ならば$\varphi(p) \neq 0$が言えればよい。

説明が難しいが、$x_i$と$x_j,y_j$が無関係だということを言えばok? (自信なし)

後半は準同型定理より、

\begin{align*}
              & R/\mathrm{Ker}{\varphi} \cong \mathrm{Im}{\varphi} \\
  \Rightarrow & R/I \cong L
\end{align*}

\section{知識}

斜体ならば可換性を課さないが、体ならば可換性がある。

体の定義は以下の通り。

\begin{itembox}[l]{ 定義(体)}
  空でない集合$K$が体(field)であるとは,

  1. $K$が単位元を持つ可換環

  2. $K$の0でない任意の元が乗法逆元を持つ,すなわち,$a \neq 0$に対し,$aa^{-1}=1$となるものが存在する。言い換えると$K^{\times}=K\setminus\{0\}$である

  の2つが成り立つことをいう。ただし,$K^{\times}$とは$K$の乗法群を指す。
\end{itembox}

この時、右イデアルと左イデアルは同じになる。

イデアルの定義は以下の通り。

\begin{itembox}[l]{ 定義(イデアル)}
  $R$を環とし,$I \subset R$とする。$I$について,

  1. $I$は加法について部分群である

  2. $r \in R, x \in I \Rightarrow rx \in I$

  3. $r \in R, x \in I \Rightarrow xr \in I$

  ...(中略)...,1,2,3が成り立つとき,両側イデアル (two-sided ideal)という。
\end{itembox}

$S (\subset R)$から生成された有限生成イデアルの一般形は以下の通り。

\begin{align*}
  (S)=\{r_1s_1+\cdots r_ns_n | r_k \in R, s_k \in S, n \geq 1 \}
\end{align*}

群の準同型定理の主張は以下の通り。

群準同型$f:G_1 \to G_2$に対して、写像$F:G_1/\mathrm{Ker}f \to \mathrm{Im}f$は群準同型であり、特に、$G_1/\mathrm{Ker}f \cong \mathrm{Im}f$である。

\section{参考文献}
\begin{thebibliography}{9}
  \bibitem{site:1}
  数学の景色.“体の定義と具体例4つ”.2022年6月13日.\url{https://mathlandscape.com/field/}
  \bibitem{site:2}
  数学の景色.“イデアル(環論)とは~定義・具体例・基本的性質の証明~”.2022年6月19日.\url{https://mathlandscape.com/ideal/}
\end{thebibliography}
\end{document}
