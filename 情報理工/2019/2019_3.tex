\documentclass[a4paper, 10pt, dvipdfmx]{jlreq}

\usepackage{amsmath,amsfonts,amssymb}
\usepackage{bm}
\usepackage{mathtools}
\usepackage{siunitx}
\usepackage[dvipdfmx]{graphicx}
\usepackage[dvipdfmx]{color}
\usepackage[dvipdfmx, colorlinks=true, allcolors=blue]{hyperref}
\usepackage{listings, jlisting}
\usepackage{tikz}
\usepackage{physics}
\usepackage{url}

\Urlmuskip=0mu plus 10mu
\allowdisplaybreaks[4]
\frenchspacing
\definecolor{OliveGreen}{rgb}{0.0,0.6,0.0}
\definecolor{Orenge}{rgb}{0.89,0.55,0}
\definecolor{SkyBlue}{rgb}{0.28, 0.28, 0.95}
\lstset{
  language={c++},
  basicstyle={\ttfamily},
  identifierstyle={\small},
  ndkeywordstyle={\small},
  frame=single,
  breaklines=true,
  numbers=left,
  xrightmargin=0zw,
  xleftmargin=3zw,
  numberstyle={\scriptsize},
  lineskip=-0.9ex,
  keywordstyle={\small\bfseries\color{SkyBlue}},  
  commentstyle={\color{OliveGreen}}, 
  stringstyle={\small\ttfamily\color{Orenge}}    
}

\begin{document}

\title{2019年度 大問1}
\author{hari64boli64 (hari64boli64@gmail.com)}
\date{\today}
\maketitle


\section{問題}

ファルカスの補題を分離定理と有限生成錐の閉性から示せ。

(個人の感想だが、この問題はかなり捨て問な気がする。)

\section{解答}

\subsection*{(1)}

これは分離定理を示す問題である。

\subsubsection*{(1-1)}

$\mathbb{R}^n$における有界な閉集合$K$はコンパクト集合であること、及び、コンパクト集合上で定義される連続関数$f:K \to \mathbb{R},f(z)=||x-z||$は、最大値と最小値を持つことから、$||x-y||=\inf_{z\in K}f(z)$を達成する$y \in K$が取れる。

なお、$K$は有界とは限らないが、本問に関しては、$x$からの距離で適当に区切っても題意に影響がない為、有界な閉集合に限定出来る。

\subsubsection*{(1-2)}

凸性より明らか。省略する。

\subsubsection*{(1-3)}

$d=0$として良い。

イメージとしては、$x$と$y$の間の垂直二等分線(分離超平面)で、$x$と$y$が別々の領域に分かれる為、内積の正負が反転するという感じである。

しかし、厳密にこれを記述することはかなり難しいと思われる。

ここでは省略する。

(「分離超平面定理」を参照)

\subsection*{(2)}

\subsubsection*{(2-1)}

$C(\mathcal{A}) \supseteq \bigcup_{\mathcal{B}}C(\mathcal{B})$は明らか。

$C(\mathcal{A}) \subseteq \bigcup_{\mathcal{B}}C(\mathcal{B})$を示す。

$C(\mathcal{A})$の元$x$を取る。

$x=\lambda_1 a_1+ \cdots + \lambda_m a_m$に関して、$a_m=\mu_1 a_1 + \cdots + \mu_{m-1} a_{m-1}$と書けたと仮定する。

全ての$1 \leq i < m$に対して、$\lambda_i+\lambda_m \mu_i \geq 0$が成立すれば、$x \in C(\{a_i | 1 \leq i < m \})$となる。

そうでない場合、$\lambda_i+k \mu_i$が最小の$k$で0になるようなインデックスを$i$に取ると、$\lambda_m=\lambda_m'+k$として、

\begin{align*}
    x & =\lambda_1 a_1 + \cdots + \lambda_i a_i + \cdots + \lambda_m a_m                                                       \\
      & =\lambda_1 a_1 + \cdots + \lambda_i a_i + \cdots +(\lambda_m'+k) a_m                                                   \\
      & =\lambda_1 a_1 + \cdots + \lambda_i a_i + \cdots + \lambda_m' a_m + k(\mu_1a_1+\cdots+\mu_{m-1}a_{m-1})                \\
      & =(k\mu_1+\lambda_1) a_1 + \cdots + (k\mu_i+\lambda_i)a_i + \cdots + (k \mu_{m-1}+\lambda_{m-1}) a_{m-1}+\lambda_m' a_m \\
      & =(k\mu_1+\lambda_1) a_1 + \cdots + 0 + \cdots + (k \mu_{m-1}+\lambda_{m-1}) a_{m-1}+\lambda_m' a_m                     \\
      & \in C(\{a_j | 1 \leq j \leq m, j\neq i \})
\end{align*}

となり、要素数を減らすことが出来る。

以下、帰納的に一次独立になるまで、この議論を繰り返せばよい。

よって、示された。

\subsubsection*{(2-2)}

$C(\mathcal{B})$が閉集合であることが言えれば、有限個の閉集合の和集合は閉集合であることから、$C(\mathcal{A})$も閉集合であると言える。

$C(\mathcal{B})$が閉集合であることを示す。

これは、有限生成錐の閉性を示せば良い。

これもかなり記述は難しい気がする。

ここでは省略する。

(「有限生成錐が閉集合になることについて」を参照)

\subsection*{(3)}

以上を基に、ファルカスの補題を示す。

\subsubsection*{(3-1)}

\begin{align*}
    (P) & \Rightarrow \exists \lambda \in \mathbb{R}^m \text{ s.t. } A\lambda =x,\lambda \geq 0
\end{align*}

この時、$c^TA \leq 0 \Rightarrow c^TA\lambda =c^Tx \leq 0$

となり、$(P)\Rightarrow \overline{(Q)}$が示された。

\subsubsection*{(3-2)}

$A$の各列ベクトルが生成する有限生成錐$C(\mathcal{A}=\{a_i\}_{1\leq i \leq m})$は非空な凸閉集合である。

$\overline{(P)}$ならば、$x \notin C(\mathcal{A})$であり、分離定理より、$\langle c,x\rangle >0$かつ、$\langle c,a_i\rangle \leq 0$となる$c$が存在する。これは$(Q)$に他ならない。

よって、$\overline{(P)}\Rightarrow (Q)$が示された。

\section{知識}

ファルカスの補題は、弱双対定理からも示せる。

\begin{align*}
    \min_{x} c^Tx \text{ s.t. } Ax = b, x \geq \bm{0}
\end{align*}

\begin{align*}
    \max_{y} b^Ty \text{ s.t. } A^Ty \leq c
\end{align*}

記号を入れ替えて、

\begin{align*}
                & \min_{\lambda} \bm{0}^T\lambda \text{ s.t. } A\lambda = x, \lambda \geq \bm{0} \\
    \Rightarrow & \min_{\lambda} 0 \text{ s.t. } A\lambda = x, \lambda \geq \bm{0}
\end{align*}

\begin{align*}
                & \max_{c} x^Tc \text{ s.t. } A^Tc \leq \bm{0} \\
    \Rightarrow & \max_{c} c^Tx \text{ s.t. } c^TA \leq \bm{0}
\end{align*}

となる。

$\overline{(Q)}$ならば、弱双対定理より、$(P)$となる。


\begin{thebibliography}{9}
    \bibitem{site:1}
    "コンパクト集合上の連続関数".\url{http://www.misojiro.t.u-tokyo.ac.jp/~murota/lect-kisosuri/contfncompactset070914.pdf}
    \bibitem{site:2}
    Wikipedia."分離超平面定理".2022年3月16日.\url{https://ja.wikipedia.org/wiki/%E5%88%86%E9%9B%A2%E8%B6%85%E5%B9%B3%E9%9D%A2%E5%AE%9A%E7%90%86}
    \bibitem{site:3}
    WIIS."凸集合どうしのミンコフスキー和(ミンコフスキー差)".2022年2月21日.\url{https://wiis.info/math/convex-analysis/convex-set/sum-of-set/}
    \bibitem{site:4}
    "Farkasの補題と線形計画法の双対定理"\url{http://www.misojiro.t.u-tokyo.ac.jp/~murota/lect-surikeikakuhou/farkas031025.pdf}
    \bibitem{site:5}
    関口 良行(東京海洋大学)."有限生成錐が閉集合になることについて".\url{https://www2.kaiyodai.ac.jp/~yoshi-s/Notes/FiniteCone.pdf}
\end{thebibliography}
\end{document}
