\documentclass[a4paper, 10pt, dvipdfmx]{jlreq}

\usepackage{diagbox}
\usepackage{amsmath,amsfonts,amssymb}
\usepackage{bm}
\usepackage{mathtools}
\usepackage{siunitx}
\usepackage[dvipdfmx]{graphicx}
\usepackage[dvipdfmx]{color}
\usepackage[dvipdfmx, colorlinks=true, allcolors=blue]{hyperref}
\usepackage{listings}
\usepackage{tikz}
\usepackage{physics}
\usepackage{url}

\Urlmuskip=0mu plus 10mu
\allowdisplaybreaks[4]
\frenchspacing
\definecolor{OliveGreen}{rgb}{0.0,0.6,0.0}
\definecolor{Orange}{rgb}{0.89,0.55,0}
\definecolor{SkyBlue}{rgb}{0.28, 0.28, 0.95}
\lstset{
  language={c++},
  basicstyle={\ttfamily},
  identifierstyle={\small},
  ndkeywordstyle={\small},
  frame=single,
  breaklines=true,
  numbers=left,
  xrightmargin=0zw,
  xleftmargin=3zw,
  numberstyle={\scriptsize},
  lineskip=-0.9ex,
  keywordstyle={\small\bfseries\color{SkyBlue}},  
  commentstyle={\color{OliveGreen}}, 
  stringstyle={\small\ttfamily\color{Orange}}    
}

\begin{document}

\title{2010年度 大問2}
\author{hari64boli64 (hari64boli64@gmail.com)}
\date{\today}
\maketitle

\section{問題}


\begin{table}[hbtp]
  \caption{Probabilities of $X$ and $Y$}
  \label{table:prob}
  \centering
  \begin{tabular}{c|cc}
    \diagbox{Y}{X} & 0(1-$\epsilon$) & 1($\epsilon$) \\
    \hline
    0              & 0.9             & 0.1           \\
    1              & 0.1             & 0.9           \\
  \end{tabular}
\end{table}

\section{解答}

\subsection*{(1)}

$P(X=1|Y=0)=\frac{P(X=1,Y=0)}{P(Y=0)}=\frac{0.1\epsilon}{0.9-0.8\epsilon}$

\subsection*{(2)}

\begin{align*}
  P(X=1|Z_n=z_n) & =\frac{P(X=1,Z_n=z_n)}{P(Z_n=z_n)}                                                                                                                 \\
                 & =\frac{\epsilon 0.9^{z_n} 0.1^{n-z_n} {}_n C_{z_n}}{\epsilon 0.9^{z_n} 0.1^{n-z_n} {}_n C_{z_n} + (1-\epsilon) 0.1^{z_n} 0.9^{n-z_n} {}_n C_{z_n}} \\
                 & =\frac{\epsilon 0.1^{n-z_n}}{\epsilon 0.1^{n-z_n} + (1-\epsilon) 0.9^{n-z_n}}
\end{align*}

\subsection*{(3)}

確率0.1で+1、確率0.9で-1されるランダムウォークを考える。

初めて+2か-2に到達するまでのステップ数の期待値が答え。

漸化式をたてて解くと、座標0には偶数回目のみ非負の確率で存在し、$n$回目だと、確率は$0.18^{n/2}$である。

この結果から、期待値は、

\begin{align*}
  \sum_{n=0,2|n}^{\infty}n 0.18^{\frac{n-2}{2}} (0.9^2+0.1^2) ={} & \frac{2\times 0.82}{0.18}\sum_{n=1}^{\infty/2}n(0.18)^n \\
  ={}                                                             & \frac{1.64}{0.18}\frac{0.18}{(1-0.18)^2}                \\
  ={}                                                             & 2.439024390243902\cdots
\end{align*}

となる。

2ステップで終了するのが確率0.82で起きるので、全体で大体2.5ステップ程度というのは、直感にも合っている。

\section{知識}

似た話として、カタラン数があるが、今回は関係なかった。

\section{おまけ}

\lstinputlisting[caption=randomWalk,label=code:randomWalk,language=Python]{2.py}

average: 2.43544

\end{document}
