\documentclass[a4paper, 10pt, dvipdfmx]{jlreq}

\usepackage{amsmath,amsfonts,amssymb}
\usepackage{bm}
\usepackage{mathtools}
\usepackage{siunitx}
\usepackage[dvipdfmx]{graphicx}
\usepackage[dvipdfmx]{color}
\usepackage[dvipdfmx, colorlinks=true, allcolors=blue]{hyperref}
\usepackage{listings}
\usepackage{tikz}
\usepackage{physics}
\usepackage{url}

\Urlmuskip=0mu plus 10mu
\allowdisplaybreaks[4]
\frenchspacing
\definecolor{OliveGreen}{rgb}{0.0,0.6,0.0}
\definecolor{Orange}{rgb}{0.89,0.55,0}
\definecolor{SkyBlue}{rgb}{0.28, 0.28, 0.95}
\lstset{
  language={c++},
  basicstyle={\ttfamily},
  identifierstyle={\small},
  ndkeywordstyle={\small},
  frame=single,
  breaklines=true,
  numbers=left,
  xrightmargin=0zw,
  xleftmargin=3zw,
  numberstyle={\scriptsize},
  lineskip=-0.9ex,
  keywordstyle={\small\bfseries\color{SkyBlue}},  
  commentstyle={\color{OliveGreen}}, 
  stringstyle={\small\ttfamily\color{Orange}}    
}

\begin{document}

\title{2012年度 大問1}
\author{hari64boli64 (hari64boli64@gmail.com)}
\date{\today}
\maketitle

\section{問題}

複素積分

\section{解答}

\subsection*{(1)}

\begin{equation*}
  \frac{1}{2^n}\sum_{\sigma_1=\pm1}\sum_{\sigma_2=\pm1}\cdots\sum_{\sigma_n=\pm1}\bm{\sigma}^\top A\bm{\sigma}=\Tr A
\end{equation*}

\subsection*{(2)}

\begin{equation*}
  \frac{1}{2^3}\sum_{\sigma_1=\pm1}\sum_{\sigma_2=\pm1}\sum_{\sigma_3=\pm1}\frac{1}{2 \pi i}\oint_{|z|=1}\bm{\sigma}^\top (zI-A)^{-1}\bm{\sigma}dz=3
\end{equation*}

\subsection*{(3)}

\begin{equation*}
  \frac{1}{2^n}\sum_{\sigma_1=\pm1}\sum_{\sigma_2=\pm1}\cdots\sum_{\sigma_n=\pm1}\frac{1}{2 \pi i}\oint_{\partial D}\bm{\sigma}^\top (zI-A)^{-1}\bm{\sigma}dz=\text{D内の固有値の個数(重複を含む)}
\end{equation*}

\section{知識}

複素積分(todo)

% \begin{thebibliography}{9}

% \end{thebibliography}

\end{document}
