\documentclass[a4paper, 10pt, dvipdfmx]{jlreq}

\usepackage{amsmath,amsfonts,amssymb}
\usepackage{bm}
\usepackage{mathtools}
\usepackage{siunitx}
\usepackage[dvipdfmx]{graphicx}
\usepackage[dvipdfmx]{color}
\usepackage[dvipdfmx, colorlinks=true, allcolors=blue]{hyperref}
\usepackage{listings}
\usepackage{tikz}
\usepackage{physics}
\usepackage{url}

\Urlmuskip=0mu plus 10mu
\allowdisplaybreaks[4]
\frenchspacing
\definecolor{OliveGreen}{rgb}{0.0,0.6,0.0}
\definecolor{Orange}{rgb}{0.89,0.55,0}
\definecolor{SkyBlue}{rgb}{0.28, 0.28, 0.95}
\lstset{
  language={c++},
  basicstyle={\ttfamily},
  identifierstyle={\small},
  ndkeywordstyle={\small},
  frame=single,
  breaklines=true,
  numbers=left,
  xrightmargin=0zw,
  xleftmargin=3zw,
  numberstyle={\scriptsize},
  lineskip=-0.9ex,
  keywordstyle={\small\bfseries\color{SkyBlue}},  
  commentstyle={\color{OliveGreen}}, 
  stringstyle={\small\ttfamily\color{Orange}}    
}

\begin{document}

\title{2017年度 大問3}
\author{hari64boli64 (hari64boli64@gmail.com)}
\date{\today}
\maketitle

\section{問題}

累積分布関数の逆関数

\section{解答}

\subsection*{(1)}

\begin{equation*}
  R[T]=\frac{(1-p)\log(1-p)+p\log{p}}{p-1}
\end{equation*}

\subsection*{(2)}

確率密度関数を$P$とする。

\begin{align*}
  \mathrm{Pr}(B) & = \int_{x \in B} P(x) \dd{x}                       \\
                 & = \int P(x) \dd{x} - \int_{x \notin B} P(x) \dd{x} \\
                 & = 1 - \int_{x \notin B} P(x) \dd{x}                \\
                 & = 1 - \int_{\int_0^x P(x) \dd{x} < p} P(x) \dd{x}  \\
                 & = 1 - p
\end{align*}

最後に、$P(x) \geq 0$から導かれる、範囲についての単調性を用いた(もう少し厳密なやり方があるかも)。

\begin{align*}
  R[X] & =\frac{1}{1-p}\int_p^1 F_X^{-1}(u) \dd{u}                                                            \\
       & =\frac{1}{1-p}\int_{u \geq p} F_X^{-1}(u) \dd{u}                                                     \\
       & =\frac{1}{1-p}\int_{F(x) \geq p} P(x)x \dd{x} \; \qty(\because F_X(x)=u, \dv{u}{x}=\dv{F_X}{x}=P(x)) \\
       & =\frac{1}{1-p}\int P(x) (xI_B) \dd{x}
\end{align*}


$x \in B \setminus A \Rightarrow x \geq F_X^{-1}(p), x \in A \setminus B \Rightarrow x < F_X^{-1}(p)$より、

\begin{align*}
  \int_{x\in B \setminus A} xP(x)dx & \leq F_X^{-1}(p)\mathrm{Pr}(B \setminus A)                                              \\
                                    & = F_X^{-1}(p)\mathrm{Pr}(A \setminus B) \; (\because \mathrm{Pr}(A)=\mathrm{Pr}(B)=1-p) \\
                                    & \leq \int_{x\in A \setminus B} xP(x)dx
\end{align*}

ただし、一つ目の不等号における等号は$\mathrm{B \setminus A}=0$の時に成立する($A=B$とは、厳密には言えない)。

\subsection*{(3)}

$A=\qty(X+Y \geq F_{X+Y}^{-1}(p)),B=\qty(X \geq F_X^{-1}(p)),C=\qty(Y \geq F_Y^{-1}(p))$とする。

$\mathrm{Pr}(A)=\mathrm{Pr}(B)=1-p$であることから、(2)後半より、$\mathrm{E}[X \cdot I_A] \leq \mathrm{E}[X \cdot I_B]$である。

同様に、$\mathrm{E}[Y \cdot I_A] \leq \mathrm{E}[Y \cdot I_C]$である。

よって、

\begin{align*}
                  & \mathrm{E}[X \cdot I_A]+ \mathrm{E}[Y \cdot I_A] \leq \mathrm{E}[X \cdot I_B] + \mathrm{E}[Y \cdot I_C]                             \\
  \Leftrightarrow & \mathrm{E}\qty[(X+Y) \cdot I_{X+Y \geq F_{X+Y}^{-1}(p)}] \leq E[X \cdot I_{X \geq F_X^{-1}(p)}] + E[Y \cdot I_{Y \geq F_Y^{-1}(p)}] \\
  \Leftrightarrow & \mathrm{R}[X+Y] \leq \mathrm{R}[X] + \mathrm{R}[Y]
\end{align*}

\end{document}
