\documentclass[a4paper, 10pt, dvipdfmx]{jlreq}

\usepackage{amsmath,amsfonts,amssymb}
\usepackage{bm}
\usepackage{mathtools}
\usepackage{siunitx}
\usepackage[dvipdfmx]{graphicx}
\usepackage[dvipdfmx]{color}
\usepackage[dvipdfmx, colorlinks=true, allcolors=blue]{hyperref}
\usepackage{listings}
\usepackage{tikz}
\usepackage{physics}
\usepackage{url}

\Urlmuskip=0mu plus 10mu
\allowdisplaybreaks[4]
\frenchspacing
\definecolor{OliveGreen}{rgb}{0.0,0.6,0.0}
\definecolor{Orange}{rgb}{0.89,0.55,0}
\definecolor{SkyBlue}{rgb}{0.28, 0.28, 0.95}
\lstset{
  language={c++},
  basicstyle={\ttfamily},
  identifierstyle={\small},
  ndkeywordstyle={\small},
  frame=single,
  breaklines=true,
  numbers=left,
  xrightmargin=0zw,
  xleftmargin=3zw,
  numberstyle={\scriptsize},
  lineskip=-0.9ex,
  keywordstyle={\small\bfseries\color{SkyBlue}},  
  commentstyle={\color{OliveGreen}}, 
  stringstyle={\small\ttfamily\color{Orange}}    
}

\begin{document}

\title{2017年度 大問5}
\author{hari64boli64 (hari64boli64@gmail.com)}
\date{\today}
\maketitle

\section{問題}

ラプラシアン行列

\subsection*{(1)}
経路数

\subsection*{(2)}
ケーリーハミルトンの定理より、背理法

\subsection*{(3)}
\begin{align*}
                  & \sum_{j=1}^{n}{0}=0                             \\
  \Leftrightarrow & \sum_{j=1}^{n}{0u_j}=0                          \\
  \Leftrightarrow & \sum_{j=1}^{n}\qty(\sum_{i=1}^{n}{L_{ij}})u_j=0 \\
  \Leftrightarrow & \sum_{i=1}^{n}\sum_{j=1}^{n}{L_{ij}u_j}=0       \\
  \Leftrightarrow & \sum_{i=1}^{n}{(Lu)_i}=0                        \\
  \Leftrightarrow & \sum_{i=1}^{n}{\lambda u_i}=0                   \\
  \Leftrightarrow & \sum_{i=1}^{n}{u_i}=0
\end{align*}

\subsection*{(4)}
\begin{align*}
  \bm{x}^\top  L \bm{x} & =\sum_{i,j}x_i L_{ij} x_j                                                                \\
                        & =\sum_{i,j}x_i (D_{ij}-A_{ij}) x_j                                                       \\
                        & =\sum_{i}x_i^2 D_{ii}-\sum_{i<j}x_i x_j A_{ij}- \sum_{i>j}x_i x_j A_{ij}                 \\
                        & =\sum_{i}\qty(x_i^2 \sum_{j}{A_{ij}})-\sum_{i<j}x_i x_j A_{ij}- \sum_{i<j}x_j x_i A_{ji} \\
                        & =\sum_{i<j}(x_i^2 -2x_i x_j +x_j^2) A_{ij}                                               \\
                        & =\sum_{i<j}a_{ij}(x_i -x_j)^2                                                            \\
                        & \geq 0
\end{align*}

\subsection*{(5)}

$\dv{\bm{x}}{t}=-L\bm{x}$より、$\bm{x}(t)=e^{-Lt}\bm{x}(0)$となる。

Lの固有値が全て非負実数の為、$\bm{\overline{x}}=\lim_{t \to \infty}\bm{x}(t) = \bm{0}$となる。

また、収束の速さは$L$の固有値に依存する。


\end{document}
