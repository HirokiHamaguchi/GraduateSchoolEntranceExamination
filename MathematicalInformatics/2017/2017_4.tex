\documentclass[a4paper, 10pt, dvipdfmx]{jlreq}

\usepackage{amsmath,amsfonts,amssymb}
\usepackage{bm}
\usepackage{mathtools}
\usepackage{siunitx}
\usepackage[dvipdfmx]{graphicx}
\usepackage[dvipdfmx]{color}
\usepackage[dvipdfmx, colorlinks=true, allcolors=blue]{hyperref}
\usepackage{listings}
\usepackage{tikz}
\usepackage{physics}
\usepackage{url}

\Urlmuskip=0mu plus 10mu
\allowdisplaybreaks[4]
\frenchspacing
\definecolor{OliveGreen}{rgb}{0.0,0.6,0.0}
\definecolor{Orange}{rgb}{0.89,0.55,0}
\definecolor{SkyBlue}{rgb}{0.28, 0.28, 0.95}
\lstset{
  language={c++},
  basicstyle={\ttfamily},
  identifierstyle={\small},
  ndkeywordstyle={\small},
  frame=single,
  breaklines=true,
  numbers=left,
  xrightmargin=0zw,
  xleftmargin=3zw,
  numberstyle={\scriptsize},
  lineskip=-0.9ex,
  keywordstyle={\small\bfseries\color{SkyBlue}},  
  commentstyle={\color{OliveGreen}}, 
  stringstyle={\small\ttfamily\color{Orange}}    
}

\begin{document}

\title{2017年度 大問4}
\author{hari64boli64 (hari64boli64@gmail.com)}
\date{\today}
\maketitle

\section{問題}

歪対称行列など

\section{解答}

\subsection*{(1)}

\begin{align*}
  [(M,X),X]      & =(M,X)X-X(M,X)     \\
                 & =MX^2+XMX-XMX-X^2M \\
                 & =MX^2-X^2M         \\
  [(M,X),X]^\top & =(MX^2-X^2M)^\top  \\
                 & =MX^2-X^2M
\end{align*}

よって、

\begin{equation*}
  \dv{(X-X^\top )}{t}=0
\end{equation*}

これは、初期状態が対称行列であることと合わせると、常に$X=X^\top $であることを示す。

\subsection*{(2)}

\begin{align*}
  (M,X)^\top & =M^\top X^\top +X^\top M^\top \\
             & =-MX-XM                       \\
             & =-(M,X)
\end{align*}

より、$(M,X)$は歪対称行列である。

よって、

\begin{align*}
  \dv{X}{t} & =\dv{Q}{t}X_0Q^\top +QX_0\dv{Q^\top }{t} \\
            & =(M,X)QX_0Q^\top +QX_0Q^\top (M,X)^\top  \\
            & =(M,X)X+X(M,X)^\top                      \\
            & =(M,X)X-X(M,X)                           \\
            & =[(M,X),X]
\end{align*}

となり、示される。

\subsection*{(3)}

$Q$が直交行列であれば、同じ固有値を持つ。

\begin{align*}
  \dv{Q^\top Q}{t} & =Q^\top (M,X)^\top Q+Q^\top (M,X)Q \\
                   & =-Q^\top (M,X)Q+Q^\top (M,X)Q      \\
                   & =0
\end{align*}

これは、初期状態が単位行列であることと合わせると、常に$Q^\top Q=I$であることを示す。

(何故か$\dv{QQ^\top }{t}=0$を示す方針は、未だ不明)

\subsection*{(4)}

\end{document}
