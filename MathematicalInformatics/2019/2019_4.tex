\documentclass[a4paper, 10pt, dvipdfmx]{jlreq}

\usepackage{amsmath,amsfonts,amssymb}
\usepackage{bm}
\usepackage{mathtools}
\usepackage{siunitx}
\usepackage[dvipdfmx]{graphicx}
\usepackage[dvipdfmx]{color}
\usepackage[dvipdfmx, colorlinks=true, allcolors=blue]{hyperref}
\usepackage{listings, jlisting}
\usepackage{tikz}
\usepackage{physics}
\usepackage{url}

\Urlmuskip=0mu plus 10mu
\allowdisplaybreaks[4]
\frenchspacing
\definecolor{OliveGreen}{rgb}{0.0,0.6,0.0}
\definecolor{Orenge}{rgb}{0.89,0.55,0}
\definecolor{SkyBlue}{rgb}{0.28, 0.28, 0.95}
\lstset{
  language={c++},
  basicstyle={\ttfamily},
  identifierstyle={\small},
  ndkeywordstyle={\small},
  frame=single,
  breaklines=true,
  numbers=left,
  xrightmargin=0zw,
  xleftmargin=3zw,
  numberstyle={\scriptsize},
  lineskip=-0.9ex,
  keywordstyle={\small\bfseries\color{SkyBlue}},  
  commentstyle={\color{OliveGreen}}, 
  stringstyle={\small\ttfamily\color{Orenge}}    
}

\begin{document}

\title{2019年度 大問1}
\author{hari64boli64 (hari64boli64@gmail.com)}
\date{\today}
\maketitle


\section{問題}

\begin{align*}
    e^X:=\sum_{k=0}^\infty \frac{1}{k!}X^k
\end{align*}

\section{解答}

\subsection*{(1)}

\begin{align*}
    ||e^X||     & =\left\lVert \sum_{k=0}^\infty \frac{1}{k!}X^k\right\rVert                 \\
                & \leq \sum_{k=0}^\infty \frac{1}{k!}\left\lVert X^k\right\rVert             \\
                & \leq \sum_{k=0}^\infty \frac{1}{k!}\left\lVert X\right\rVert^k             \\
                & =e^{||X||}                                                                 \\
    ||e^X-I||   & =\left\lVert X^0+ \sum_{k=1}^\infty \frac{1}{k!}X^k -I\right\rVert         \\
                & =\left\lVert \sum_{k=1}^\infty \frac{1}{k!}X^k\right\rVert                 \\
                & =\left\lVert X\sum_{k=0}^\infty \frac{1}{(k+1)!}X^k\right\rVert            \\
                & \leq ||X||\left( \sum_{k=0}^{\infty} \frac{1}{(k+1)!} ||X||^k \right)      \\
                & \leq ||X||\left( \sum_{k=0}^{\infty} \frac{1}{k!} ||X||^k \right)          \\
                & =||X||e^{||X||}                                                            \\
    ||e^X-I-X|| & =\left\lVert X^0+X^1 +\sum_{k=2}^\infty \frac{1}{k!}X^k -I - X\right\rVert \\
                & =\left\lVert \sum_{k=2}^\infty \frac{1}{k!}X^k\right\rVert                 \\
                & =\left\lVert X^2 \sum_{k=0}^\infty \frac{1}{(k+2)!}X^k\right\rVert         \\
                & \leq ||X||^2\left( \sum_{k=0}^{\infty} \frac{1}{(k+2)!} ||X||^k \right)    \\
                & \leq ||X||^2\left( \sum_{k=0}^{\infty} \frac{1}{k!} ||X||^k \right)        \\
                & =||X||e^{||X||}                                                            \\
\end{align*}

\subsection*{(2)}

まず、一般の$0\leq i < m$に対し、以下が成立する。

\begin{align*}
    \left\lVert P^iQ^{m-1-i}\right\rVert & = \left\lVert e^{\frac{i(A+B)}{m}}e^{\frac{(m-1-i)A}{m}}e^{\frac{(m-1-i)B}{m}} \right\rVert                                                                                \\
                                         & \leq e^{\left\lVert\frac{i(A+B)}{m}\right\rVert}e^{\left\lVert\frac{(m-1-i)A}{m}\right\rVert}e^{\left\lVert\frac{(m-1-i)B}{m}\right\rVert}                                 \\
                                         & \leq e^{\left\lVert\frac{iA}{m}\right\rVert+\left\lVert\frac{iB}{m}\right\rVert}e^{\left\lVert\frac{(m-1-i)A}{m}\right\rVert}e^{\left\lVert\frac{(m-1-i)B}{m}\right\rVert} \\
                                         & = e^{\left\lVert\frac{iA}{m}\right\rVert}e^{\left\lVert\frac{iB}{m}\right\rVert}e^{\left\lVert\frac{(m-1-i)A}{m}\right\rVert}e^{\left\lVert\frac{(m-1-i)B}{m}\right\rVert} \\
                                         & = e^{\left\lVert\frac{iA}{m}\right\rVert+\left\lVert\frac{(m-1-i)A}{m}\right\rVert}e^{\left\lVert\frac{iB}{m}\right\rVert+\left\lVert\frac{(m-1-i)B}{m}\right\rVert}       \\
                                         & \leq e^{\frac{m-1}{m}||A||}e^{\frac{m-1}{m}||B||}                                                                                                                          \\
                                         & = e^{\frac{m-1}{m}(||A||+||B||)}                                                                                                                                           \\
\end{align*}

ただし、途中で$e$の単調性つまり、$x \leq y\Rightarrow e^x \leq e^y$を用いた。

なお、$P \neq Q$という事には十分に注意する必要がある。安易な指数法則は適用出来ない。

以上より、

\begin{align*}
    ||P^m-Q^m|| & =\left\lVert \sum_{i=0}^{m-1}P^i(P-Q)Q^{m-1-i}\right\rVert       \\
                & \leq ||P-Q||\left\lVert \sum_{i=0}^{m-1}P^iQ^{m-1-i}\right\rVert \\
                & \leq ||P-Q||m e^{\frac{m-1}{m}(||A||+||B||)}
\end{align*}

となる。

\subsection*{(3)}

$f(X):=e^X-I,g(X):=e^X-I-X$である。

よって、

\begin{align*}
      & g\left(\frac{A+B}{m}\right)-g\left(\frac{A}{m}\right)-g\left(\frac{B}{m}\right)-f\left(\frac{A}{m}\right)f\left(\frac{B}{m}\right)                                                                         \\
    = & \left(e^{\frac{A+B}{m}}-I-\frac{A+B}{m}\right)-\left(e^{\frac{A}{m}}-I-\frac{A}{m}\right)-\left(e^{\frac{B}{m}}-I-\frac{B}{m}\right)-\left(e^{\frac{A}{m}}-I\right)\left(e^{\frac{B}{m}}-I\right)          \\
    = & \left(e^{\frac{A+B}{m}}-\frac{A+B}{m}\right)-\left(e^{\frac{A}{m}}-\frac{A}{m}\right)-\left(e^{\frac{B}{m}}-\frac{B}{m}\right)-\left(e^{\frac{A}{m}}e^{\frac{B}{m}}-e^{\frac{A}{m}}-e^{\frac{B}{m}}\right) \\
    = & e^{\frac{A+B}{m}}-e^{\frac{A}{m}}e^{\frac{B}{m}}                                                                                                                                                           \\
    = & P-Q
\end{align*}

となる。

\subsection*{(4)}

一部緩めの不等式評価になっていることに注意する必要があるが、以下のような式が成立する。

\begin{align*}
    ||P-Q|| & = \left\lVert g\left(\frac{A+B}{m}\right)-g\left(\frac{A}{m}\right)-g\left(\frac{B}{m}\right)-f\left(\frac{A}{m}\right)f\left(\frac{B}{m}\right) \right\rVert                                                                                                                                      \\
            & \leq \left\lVert g\left(\frac{A+B}{m}\right)\right\rVert+\left\lVert g\left(\frac{A}{m}\right)\right\rVert+\left\lVert g\left(\frac{B}{m}\right)\right\rVert+\left\lVert f\left(\frac{A}{m}\right)f\left(\frac{B}{m}\right) \right\rVert                                                           \\
            & \leq \left\lVert \frac{A+B}{m}\right\rVert^2 e^{\frac{||A+B||}{m}}+\left\lVert \frac{A}{m}\right\rVert^2 e^{\frac{||A||}{m}}+\left\lVert \frac{B}{m}\right\rVert^2 e^{\frac{||B||}{m}}+\left\lVert \frac{A}{m}\right\rVert\left\lVert \frac{B}{m} \right\rVert e^{\frac{||A||}{m}+\frac{||B||}{m}} \\
            & = \frac{1}{m^2} \left(||A+B||^2 e^{\frac{||A+B||}{m}}+ ||A||^2 e^{\frac{||A||}{m}}+ ||B||^2 e^{\frac{||B||}{m}}+ ||A|| ||B|| e^{\frac{||A||}{m}+\frac{||B||}{m}}\right)                                                                                                                            \\
            & \leq \frac{1}{m^2} \left(||A+B||^2 e^{\frac{||A||+||B||}{m}}+ ||A||^2 e^{\frac{||A||+||B||}{m}}+ ||B||^2 e^{\frac{||A||+||B||}{m}}+ ||A|| ||B|| e^{\frac{||A||+||B||}{m}}\right)                                                                                                                   \\
            & = \frac{1}{m^2}e^{\frac{||A||+||B||}{m}} \left(||A+B||^2 + ||A||^2 + ||B||^2 + ||A|| ||B|| \right)                                                                                                                                                                                                 \\
            & \leq \frac{1}{m^2}e^{\frac{||A||+||B||}{m}} \left(||A||^2+2||A||||B||+||B||^2 + ||A||^2 + ||B||^2 + ||A||||B|| \right)                                                                                                                                                                             \\
            & = \frac{1}{m^2}e^{\frac{||A||+||B||}{m}} \left(2||A||^2+3||A||||B||+2||B||^2 \right)                                                                                                                                                                                                               \\
            & \leq \frac{1}{m^2}e^{\frac{||A||+||B||}{m}} \left(2||A||^2+4||A||||B||+2||B||^2 \right)                                                                                                                                                                                                            \\
            & = \frac{2}{m^2}e^{\frac{||A||+||B||}{m}} \left(||A||+||B||\right)^2                                                                                                                                                                                                                                \\
\end{align*}

よって、

\begin{align*}
    ||P^m-Q^m|| & \leq ||P-Q||m e^{\frac{m-1}{m}(||A||+||B||)}                                                            \\
                & \leq \frac{2}{m^2}e^{\frac{||A||+||B||}{m}} \left(||A||+||B||\right)^2 m e^{\frac{m-1}{m}(||A||+||B||)} \\
                & = \frac{2}{m} \left(||A||+||B||\right)^2 e^{||A||+||B||}                                                \\
\end{align*}

が示される。

\subsection*{(4)}

常微分方程式を解いて、

\begin{align*}
    x(t)=ve^{(A+B)t}
\end{align*}

を得る。

特に、

\begin{align*}
    x(1)=ve^{A+B}=v P^m
\end{align*}

である。

一方、

\begin{align*}
    \tilde{x}^{2m}=v\left(e^{\frac{A}{m}}e^{\frac{B}{m}}\right)^m=vQ^m
\end{align*}

である。

よって、

\begin{align*}
    ||x(1)-\tilde{x}^{2m}|| & =||vP^m-vQ^m||                                                   \\
                            & \leq ||v||||P^m-Q^m||                                            \\
                            & \leq ||v||\frac{2}{m} \left(||A||+||B||\right)^2 e^{||A||+||B||} \\
                            & =\frac{C}{m}
\end{align*}

となる。ただし、$C$は$m$に依存しない定数である。

よって、

\begin{align*}
    \lim_{m \to \infty} m^\alpha ||x(1)-\tilde{x}^{2m}|| & \leq \lim_{m \to \infty} {m^\alpha \frac{C}{m}} \\
                                                         & =\lim_{m \to \infty} \frac{C}{m^{1-\alpha}}     \\
                                                         & =0
\end{align*}

となるので、

\begin{align*}
    \lim_{m \to \infty} m^\alpha ||x(1)-\tilde{x}^{2m}||=0
\end{align*}

が示される。

\section{知識}

連立1階線形常微分方程式の解は、行列の指数関数を用いて表す事が出来る。


\begin{thebibliography}{9}
    \bibitem{site:1}
    武内修.“線形代数II/連立線形微分方程式”.筑波大.2022年6月7日.\url{https://dora.bk.tsukuba.ac.jp/~takeuchi/?%E7%B7%9A%E5%BD%A2%E4%BB%A3%E6%95%B0II%2F%E9%80%A3%E7%AB%8B%E7%B7%9A%E5%BD%A2%E5%BE%AE%E5%88%86%E6%96%B9%E7%A8%8B%E5%BC%8F#j6598f91}
    \bibitem{site:2}
    \url{https://dora.bk.tsukuba.ac.jp/~takeuchi/?%E7%B7%9A%E5%BD%A2%E4%BB%A3%E6%95%B0II%2F%E9%80%A3%E7%AB%8B%E7%B7%9A%E5%BD%A2%E5%BE%AE%E5%88%86%E6%96%B9%E7%A8%8B%E5%BC%8F#j6598f91}
\end{thebibliography}
\end{document}
