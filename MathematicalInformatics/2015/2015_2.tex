\documentclass[a4paper, 10pt, dvipdfmx]{jlreq}

\usepackage{amsmath,amsfonts,amssymb}
\usepackage{bm}
\usepackage{mathtools}
\usepackage{siunitx}
\usepackage[dvipdfmx]{graphicx}
\usepackage[dvipdfmx]{color}
\usepackage[dvipdfmx, colorlinks=true, allcolors=blue]{hyperref}
\usepackage{listings}
\usepackage{tikz}
\usepackage{physics}
\usepackage{url}

\Urlmuskip=0mu plus 10mu
\allowdisplaybreaks[4]
\frenchspacing
\definecolor{OliveGreen}{rgb}{0.0,0.6,0.0}
\definecolor{Orenge}{rgb}{0.89,0.55,0}
\definecolor{SkyBlue}{rgb}{0.28, 0.28, 0.95}
\lstset{
  language={c++},
  basicstyle={\ttfamily},
  identifierstyle={\small},
  ndkeywordstyle={\small},
  frame=single,
  breaklines=true,
  numbers=left,
  xrightmargin=0zw,
  xleftmargin=3zw,
  numberstyle={\scriptsize},
  lineskip=-0.9ex,
  keywordstyle={\small\bfseries\color{SkyBlue}},  
  commentstyle={\color{OliveGreen}}, 
  stringstyle={\small\ttfamily\color{Orenge}}    
}

\begin{document}

\title{2015年度 大問2}
\author{hari64boli64 (hari64boli64@gmail.com)}
\date{\today}
\maketitle

\section{問題}

線形単回帰

\section{解答}

 (1)だけ難しいので(というか悪問?)、解答を記しておく。

\subsection*{(1)}

恐らく、この問題の想定解は不等式だと思われる。$r_{xy}r_{yz} = r_{xz}$などは不成立。
等式ではないのだろう(という話にSlackでなった)。

参考文献\cite{label:1}が本質的で、それの多次元版である。
参考文献\cite{label:2}も直感的な理解に役立つ。

恐らく色々な不等式が考えられると思うが、
その中でも特に簡単と思われるのが、以下の等式である。
$2\leq n \leq 10$に関して、それぞれ1000ケースについて成立することを確認した。
\begin{align*}
   & \theta_{\text{min}} = \arccos(r_{yz})-\arccos(r_{xy})             \\
   & \theta_{\text{max}} = \arccos(r_{yz})+\arccos(r_{xy})             \\
   & \theta_{\text{min}} \leq \arccos(r_{xz}) \leq \theta_{\text{max}}
\end{align*}
これは、相関係数を角度と見なした時の、三角不等式のようなものである。

尤も、証明まではしていない。間違っている可能性も十分にある。

\lstinputlisting[caption=2,label=code:2,language=Python]{2.py}

\begin{thebibliography}{9}
  \bibitem{label:1}
  Chappers,egreif1. ``Correlation between three variables question''. math stack exchange. 2018/5/27.\\\url{https://math.stackexchange.com/questions/284877/correlation-between-three-variables-question}
  \bibitem{label:2}
  ``相関係数を視覚化する''. 統計web. 2017/08/14.\\\url{https://bellcurve.jp/statistics/blog/14116.html}
\end{thebibliography}

\end{document}
